\section{Sigmoide}
Dada la siguiente tabla con dos clases A y B:

% Please add the following required packages to your document preamble:
% \usepackage[table,xcdraw]{xcolor}
% If you use beamer only pass "xcolor=table" option, i.e. \documentclass[xcolor=table]{beamer}
\begin{table}[!htb]
    \centering
    \begin{tabular}{|
    >{\columncolor[HTML]{3166FF}}l |l|l|l|l|l|l|l|l|l|l|l|l|}
    \hline
    {\color[HTML]{FFFFFF} 12} &  &  &  &  &  &  &  &  &  &  &  &  \\ \hline
    {\color[HTML]{FFFFFF} 11} &  &  &  &  &  &  &  &  &  &  &  &  \\ \hline
    {\color[HTML]{FFFFFF} 10} &  &  &  & \cellcolor[HTML]{FE0000}B &  &  & \cellcolor[HTML]{FE0000}B &  &  &  &  &  \\ \hline
    {\color[HTML]{FFFFFF} 9} &  &  &  &  &  &  &  &  &  &  &  &  \\ \hline
    {\color[HTML]{FFFFFF} 8} &  &  &  &  &  &  &  &  & \cellcolor[HTML]{FE0000}B &  &  &  \\ \hline
    {\color[HTML]{FFFFFF} 7} &  &  &  &  &  &  &  &  &  &  &  &  \\ \hline
    {\color[HTML]{FFFFFF} 6} &  & \cellcolor[HTML]{34FF34}A &  &  &  &  &  &  &  &  &  &  \\ \hline
    {\color[HTML]{FFFFFF} 5} &  &  &  &  &  &  &  &  &  &  &  &  \\ \hline
    {\color[HTML]{FFFFFF} 4} &  &  &  & \cellcolor[HTML]{34FF34}A &  &  &  &  &  &  &  &  \\ \hline
    {\color[HTML]{FFFFFF} 3} &  &  &  &  &  & \cellcolor[HTML]{34FF34}A &  &  &  &  &  &  \\ \hline
    {\color[HTML]{FFFFFF} 2} &  &  &  &  &  &  &  &  &  &  &  &  \\ \hline
    {\color[HTML]{FFFFFF} 1} &  &  &  &  &  &  &  &  &  &  &  &  \\ \hline
    {\color[HTML]{FFFFFF} 0} & \cellcolor[HTML]{3166FF}{\color[HTML]{FFFFFF} 1} & \cellcolor[HTML]{3166FF}{\color[HTML]{FFFFFF} 2} & \cellcolor[HTML]{3166FF}{\color[HTML]{FFFFFF} 3} & \cellcolor[HTML]{3166FF}{\color[HTML]{FFFFFF} 4} & \cellcolor[HTML]{3166FF}{\color[HTML]{FFFFFF} 5} & \cellcolor[HTML]{3166FF}{\color[HTML]{FFFFFF} 6} & \cellcolor[HTML]{3166FF}{\color[HTML]{FFFFFF} 7} & \cellcolor[HTML]{3166FF}{\color[HTML]{FFFFFF} 8} & \cellcolor[HTML]{3166FF}{\color[HTML]{FFFFFF} 9} & \cellcolor[HTML]{3166FF}{\color[HTML]{FFFFFF} 10} & \cellcolor[HTML]{3166FF}{\color[HTML]{FFFFFF} 11} & \cellcolor[HTML]{3166FF}{\color[HTML]{FFFFFF} 12} \\ \hline
    \end{tabular}
\end{table}

\begin{itemize}
    \item a)	Entrene un perceptrón con función de activación sigmoidal y con bias mediante la regla DELTA adaptada. Proponga una vector de pesos iniciales $W_0=(w_1,w_2,w_3 )^T$ y un $\alpha$. Muestre la línea inicial, el vector de pesos final, así como la línea de separación final.
	\item Use el perceptrón entrenado para determinar la clase A o B de los puntos con coordenadas (5,5) y (6,8).
\end{itemize}

Con base a lo aprendido en el ejercicio 2, no hubo detalle alguno sobre el acomodo de los datos. 
\\
Por otro lado, he notado como el uso de las funciones de activación y las reglas aplicadas, por sencillas que puedan ser, la diferencia es notable, tanto en el uso de recursos y el tiempo en resolver el problema, ya que, hasta cierto punto, a un mayor número de épocas el cambio en los pesos de la neurona ya no hay cambios sustanciales.

\newpage
\subsection{Desarrollo de Sigmoide}
\begin{tcolorbox}[breakable, size=fbox, boxrule=1pt, pad at break*=1mm,colback=cellbackground, colframe=cellborder]
\prompt{In}{incolor}{1}{\boxspacing}
\begin{Verbatim}[commandchars=\\\{\}]
\PY{k+kn}{import} \PY{n+nn}{numpy} \PY{k}{as} \PY{n+nn}{np}
\PY{k+kn}{import} \PY{n+nn}{matplotlib}\PY{n+nn}{.}\PY{n+nn}{pyplot} \PY{k}{as} \PY{n+nn}{plt}
\PY{k+kn}{import} \PY{n+nn}{time}
\end{Verbatim}
\end{tcolorbox}

    \begin{tcolorbox}[breakable, size=fbox, boxrule=1pt, pad at break*=1mm,colback=cellbackground, colframe=cellborder]
\prompt{In}{incolor}{2}{\boxspacing}
\begin{Verbatim}[commandchars=\\\{\}]
\PY{k}{def} \PY{n+nf}{CostFunction}\PY{p}{(}\PY{n}{t}\PY{p}{,} \PY{n}{y}\PY{p}{)}\PY{p}{:}
    
    \PY{n}{cost} \PY{o}{=} \PY{n}{np}\PY{o}{.}\PY{n}{sum}\PY{p}{(}\PY{l+m+mf}{0.5} \PY{o}{*} \PY{p}{(}\PY{n}{t}\PY{o}{\PYZhy{}}\PY{n}{y}\PY{p}{)} \PY{o}{*}\PY{o}{*} \PY{l+m+mi}{2}\PY{p}{)}
    \PY{c+c1}{\PYZsh{} cost = 0.5 * sum ((t\PYZhy{}y) ** 2)}
    \PY{k}{return} \PY{n}{cost}

\PY{c+c1}{\PYZsh{} Funcion Sigmoide}
\PY{k}{def} \PY{n+nf}{sigmoid}\PY{p}{(}\PY{n}{g}\PY{p}{)}\PY{p}{:}
    \PY{n}{a} \PY{o}{=} \PY{l+m+mi}{1} \PY{o}{/} \PY{p}{(}\PY{l+m+mi}{1} \PY{o}{+} \PY{n}{np}\PY{o}{.}\PY{n}{exp}\PY{p}{(}\PY{o}{\PYZhy{}}\PY{n}{g}\PY{p}{)}\PY{p}{)}
    \PY{k}{return}\PY{p}{(}\PY{n}{a}\PY{p}{)}


\PY{n}{err\PYZus{}vectorS}  \PY{o}{=} \PY{p}{[}\PY{p}{]}
\PY{n}{w\PYZus{}fS} \PY{o}{=} \PY{p}{[}\PY{p}{]}

\PY{c+c1}{\PYZsh{}\PYZpc{}\PYZpc{} Entradas}
\PY{n}{x} \PY{o}{=} \PY{n}{np}\PY{o}{.}\PY{n}{array}\PY{p}{(}\PY{p}{[}\PY{p}{[}\PY{l+m+mi}{2}\PY{p}{,}\PY{l+m+mi}{6}\PY{p}{]}\PY{p}{,}\PY{p}{[}\PY{l+m+mi}{4}\PY{p}{,}\PY{l+m+mi}{4}\PY{p}{]}\PY{p}{,}\PY{p}{[}\PY{l+m+mi}{6}\PY{p}{,}\PY{l+m+mi}{3}\PY{p}{]}\PY{p}{,}\PY{p}{[}\PY{l+m+mi}{4}\PY{p}{,}\PY{l+m+mi}{10}\PY{p}{]}\PY{p}{,}\PY{p}{[}\PY{l+m+mi}{7}\PY{p}{,}\PY{l+m+mi}{10}\PY{p}{]}\PY{p}{,}\PY{p}{[}\PY{l+m+mi}{9}\PY{p}{,}\PY{l+m+mi}{8}\PY{p}{]}\PY{p}{]}\PY{p}{)}

\PY{c+c1}{\PYZsh{}\PYZpc{}\PYZpc{} Target sigmoide}
\PY{n}{tS} \PY{o}{=} \PY{n}{np}\PY{o}{.}\PY{n}{array}\PY{p}{(}\PY{p}{[}\PY{p}{[}\PY{l+m+mi}{1}\PY{p}{]}\PY{p}{,}\PY{p}{[}\PY{l+m+mi}{1}\PY{p}{]}\PY{p}{,}\PY{p}{[}\PY{l+m+mi}{1}\PY{p}{]}\PY{p}{,}\PY{p}{[}\PY{o}{\PYZhy{}}\PY{l+m+mi}{1}\PY{p}{]}\PY{p}{,}\PY{p}{[}\PY{o}{\PYZhy{}}\PY{l+m+mi}{1}\PY{p}{]}\PY{p}{,}\PY{p}{[}\PY{o}{\PYZhy{}}\PY{l+m+mi}{1}\PY{p}{]}\PY{p}{]}\PY{p}{)}

\PY{c+c1}{\PYZsh{}\PYZpc{}\PYZpc{} Vector de pesos para sigmoide}
\PY{n}{w\PYZus{}iS} \PY{o}{=} \PY{n}{np}\PY{o}{.}\PY{n}{array}\PY{p}{(}\PY{p}{[}\PY{p}{[}\PY{l+m+mf}{0.5}\PY{p}{]}\PY{p}{,}\PY{p}{[}\PY{o}{\PYZhy{}}\PY{l+m+mf}{0.5}\PY{p}{]}\PY{p}{,}\PY{p}{[}\PY{o}{\PYZhy{}}\PY{l+m+mi}{1}\PY{p}{]}\PY{p}{]}\PY{p}{)}

\PY{n}{alpha} \PY{o}{=} \PY{l+m+mf}{0.4}

\PY{c+c1}{\PYZsh{}\PYZpc{}\PYZpc{} bias}
\PY{n}{bias} \PY{o}{=}\PY{n}{np}\PY{o}{.}\PY{n}{shape}\PY{p}{(}\PY{n}{x}\PY{p}{)}\PY{p}{[}\PY{l+m+mi}{0}\PY{p}{]}
\PY{n}{bias} \PY{o}{=} \PY{o}{\PYZhy{}}\PY{l+m+mi}{1} \PY{o}{*} \PY{n}{np}\PY{o}{.}\PY{n}{ones}\PY{p}{(}\PY{p}{(}\PY{n}{bias}\PY{p}{,}\PY{l+m+mi}{1}\PY{p}{)}\PY{p}{)}

\PY{c+c1}{\PYZsh{}\PYZpc{}\PYZpc{} Vector aumentado}
\PY{n}{xS} \PY{o}{=} \PY{n}{np}\PY{o}{.}\PY{n}{concatenate}\PY{p}{(}\PY{p}{[}\PY{n}{x}\PY{p}{,}\PY{n}{bias}\PY{p}{]}\PY{p}{,} \PY{n}{axis} \PY{o}{=} \PY{l+m+mi}{1}\PY{p}{)}
\end{Verbatim}
\end{tcolorbox}

    \begin{tcolorbox}[breakable, size=fbox, boxrule=1pt, pad at break*=1mm,colback=cellbackground, colframe=cellborder]
\prompt{In}{incolor}{3}{\boxspacing}
\begin{Verbatim}[commandchars=\\\{\}]
\PY{n}{ticS} \PY{o}{=} \PY{n}{time}\PY{o}{.}\PY{n}{time}\PY{p}{(}\PY{p}{)}

\PY{c+c1}{\PYZsh{} Dot product}
\PY{n}{a} \PY{o}{=} \PY{n}{np}\PY{o}{.}\PY{n}{dot}\PY{p}{(}\PY{n}{xS}\PY{p}{,}\PY{n}{w\PYZus{}iS}\PY{p}{)}

\PY{c+c1}{\PYZsh{}Funcion sigmoide}
\PY{n}{y} \PY{o}{=} \PY{n}{sigmoid}\PY{p}{(}\PY{n}{a}\PY{p}{)}

\PY{c+c1}{\PYZsh{}Error cuadratico medio}
\PY{n}{err} \PY{o}{=} \PY{n}{CostFunction}\PY{p}{(}\PY{n}{tS}\PY{p}{,} \PY{n}{y}\PY{p}{)}
\PY{n}{err\PYZus{}vectorS}\PY{o}{.}\PY{n}{append}\PY{p}{(}\PY{n}{err}\PY{p}{)}
\PY{n}{epochS} \PY{o}{=} \PY{l+m+mi}{0}
\PY{n}{epocas} \PY{o}{=} \PY{l+m+mi}{3000}
\end{Verbatim}
\end{tcolorbox}

    \begin{tcolorbox}[breakable, size=fbox, boxrule=1pt, pad at break*=1mm,colback=cellbackground, colframe=cellborder]
\prompt{In}{incolor}{4}{\boxspacing}
\begin{Verbatim}[commandchars=\\\{\}]
\PY{c+c1}{\PYZsh{}\PYZsh{}\PYZsh{}\PYZsh{}\PYZsh{}Entrenamiento}
\PY{k}{for} \PY{n}{i} \PY{o+ow}{in} \PY{n+nb}{range}\PY{p}{(}\PY{n}{epocas}\PY{p}{)}\PY{p}{:}
    \PY{n}{epochS}\PY{o}{+}\PY{o}{=}\PY{l+m+mi}{1}
    \PY{k}{for} \PY{n}{i} \PY{o+ow}{in} \PY{n+nb}{range}\PY{p}{(}\PY{n}{np}\PY{o}{.}\PY{n}{shape}\PY{p}{(}\PY{n}{xS}\PY{p}{)}\PY{p}{[}\PY{l+m+mi}{0}\PY{p}{]}\PY{p}{)}\PY{p}{:}
        \PY{c+c1}{\PYZsh{} Dot product}
        \PY{n}{a} \PY{o}{=} \PY{n}{np}\PY{o}{.}\PY{n}{dot}\PY{p}{(}\PY{n}{xS}\PY{p}{[}\PY{n}{i}\PY{p}{]} \PY{p}{,} \PY{n}{w\PYZus{}iS}\PY{p}{)}
        \PY{c+c1}{\PYZsh{} Hard limit}
        \PY{n}{y}\PY{p}{[}\PY{n}{i}\PY{p}{]} \PY{o}{=} \PY{n}{sigmoid}\PY{p}{(}\PY{n}{a}\PY{p}{)}
        \PY{c+c1}{\PYZsh{} Actualizacion de pesos}
        \PY{n}{x\PYZus{}T} \PY{o}{=} \PY{n}{np}\PY{o}{.}\PY{n}{reshape}\PY{p}{(}\PY{n}{xS}\PY{p}{[}\PY{n}{i}\PY{p}{]}\PY{p}{,} \PY{p}{(}\PY{n+nb}{len}\PY{p}{(}\PY{n}{w\PYZus{}iS}\PY{p}{)}\PY{p}{,}\PY{l+m+mi}{1}\PY{p}{)}\PY{p}{)}\PY{c+c1}{\PYZsh{}transpuesta}
        \PY{n}{w\PYZus{}nS} \PY{o}{=} \PY{n}{w\PYZus{}iS} \PY{o}{\PYZhy{}} \PY{n}{alpha} \PY{o}{*} \PY{n}{x\PYZus{}T} \PY{o}{*} \PY{n}{y}\PY{p}{[}\PY{n}{i}\PY{p}{]} \PY{o}{*} \PY{p}{(}\PY{n}{y}\PY{p}{[}\PY{n}{i}\PY{p}{]} \PY{o}{\PYZhy{}} \PY{n}{tS}\PY{p}{[}\PY{n}{i}\PY{p}{]}\PY{p}{)} \PY{o}{*} \PY{p}{(}\PY{l+m+mi}{1} \PY{o}{\PYZhy{}} \PY{n}{y}\PY{p}{[}\PY{n}{i}\PY{p}{]}\PY{p}{)}
        \PY{n}{w\PYZus{}iS} \PY{o}{=} \PY{n}{w\PYZus{}nS}
        
    \PY{n}{err} \PY{o}{=} \PY{l+m+mf}{0.25} \PY{o}{*} \PY{n}{CostFunction}\PY{p}{(}\PY{n}{tS}\PY{p}{,} \PY{n}{y}\PY{p}{)}
    
    \PY{n}{err\PYZus{}vectorS}\PY{o}{.}\PY{n}{append}\PY{p}{(}\PY{n}{err}\PY{p}{)}
    \PY{n}{w\PYZus{}fS}\PY{o}{.}\PY{n}{append}\PY{p}{(}\PY{n}{w\PYZus{}iS}\PY{p}{)}
    
\PY{n}{tocS} \PY{o}{=} \PY{n}{time}\PY{o}{.}\PY{n}{time}\PY{p}{(}\PY{p}{)}
\end{Verbatim}
\end{tcolorbox}

    \begin{tcolorbox}[breakable, size=fbox, boxrule=1pt, pad at break*=1mm,colback=cellbackground, colframe=cellborder]
\prompt{In}{incolor}{5}{\boxspacing}
\begin{Verbatim}[commandchars=\\\{\}]
\PY{n}{plt}\PY{o}{.}\PY{n}{figure}\PY{p}{(}\PY{l+m+mi}{0}\PY{p}{)}
\PY{n}{plt}\PY{o}{.}\PY{n}{plot}\PY{p}{(}\PY{n}{err\PYZus{}vectorS}\PY{p}{,} \PY{n}{linewidth} \PY{o}{=} \PY{l+m+mi}{2}\PY{p}{)}
\PY{n}{plt}\PY{o}{.}\PY{n}{title}\PY{p}{(}\PY{l+s+s1}{\PYZsq{}}\PY{l+s+s1}{Gráfica de Error: NEURONA SIGMOIDE}\PY{l+s+s1}{\PYZsq{}}\PY{p}{)}
\PY{n}{plt}\PY{o}{.}\PY{n}{ylabel}\PY{p}{(}\PY{l+s+s1}{\PYZsq{}}\PY{l+s+s1}{Magnitud de Error}\PY{l+s+s1}{\PYZsq{}}\PY{p}{)}
\PY{n}{plt}\PY{o}{.}\PY{n}{xlabel}\PY{p}{(}\PY{l+s+s1}{\PYZsq{}}\PY{l+s+s1}{Epócas}\PY{l+s+s1}{\PYZsq{}}\PY{p}{)}
\PY{n}{plt}\PY{o}{.}\PY{n}{scatter}\PY{p}{(}\PY{n+nb}{len}\PY{p}{(}\PY{n}{err\PYZus{}vectorS}\PY{p}{)} \PY{o}{\PYZhy{}} \PY{l+m+mi}{1}\PY{p}{,} \PY{l+m+mi}{0}\PY{p}{,} \PY{n}{color} \PY{o}{=} \PY{l+s+s1}{\PYZsq{}}\PY{l+s+s1}{r}\PY{l+s+s1}{\PYZsq{}}\PY{p}{,} \PY{n}{s} \PY{o}{=} \PY{l+m+mi}{200}\PY{p}{,} \PY{n}{marker} \PY{o}{=} \PY{l+s+s1}{\PYZsq{}}\PY{l+s+s1}{o}\PY{l+s+s1}{\PYZsq{}}\PY{p}{,} \PY{n}{alpha} \PY{o}{=} \PY{l+m+mf}{0.4}\PY{p}{)}
\PY{n}{plt}\PY{o}{.}\PY{n}{show}\PY{p}{(}\PY{p}{)}
\end{Verbatim}
\end{tcolorbox}

    \begin{center}
    \adjustimage{max size={0.9\linewidth}{0.9\paperheight}}{sections/sigmoide/output_4_0.png}
    \end{center}
    { \hspace*{\fill} \\}
    
    \begin{tcolorbox}[breakable, size=fbox, boxrule=1pt, pad at break*=1mm,colback=cellbackground, colframe=cellborder]
\prompt{In}{incolor}{6}{\boxspacing}
\begin{Verbatim}[commandchars=\\\{\}]
\PY{c+c1}{\PYZsh{}\PYZsh{}\PYZsh{}\PYZsh{}\PYZsh{}\PYZsh{}\PYZsh{}\PYZsh{}\PYZsh{}\PYZsh{}\PYZsh{}\PYZsh{}\PYZsh{}\PYZsh{}\PYZsh{}\PYZsh{}\PYZpc{}\PYZpc{} Grafica frontera de decision}
\PY{n}{fig} \PY{o}{=} \PY{n}{plt}\PY{o}{.}\PY{n}{figure}\PY{p}{(}\PY{l+m+mi}{1}\PY{p}{)}
\PY{n}{plt}\PY{o}{.}\PY{n}{xlim}\PY{p}{(}\PY{p}{[}\PY{o}{\PYZhy{}}\PY{l+m+mf}{1.0}\PY{p}{,}\PY{l+m+mf}{15.0}\PY{p}{]}\PY{p}{)}
\PY{n}{plt}\PY{o}{.}\PY{n}{ylim}\PY{p}{(}\PY{p}{[}\PY{o}{\PYZhy{}}\PY{l+m+mf}{1.0}\PY{p}{,}\PY{l+m+mf}{15.0}\PY{p}{]}\PY{p}{)}

\PY{n}{patterns} \PY{o}{=} \PY{n}{np}\PY{o}{.}\PY{n}{unique}\PY{p}{(}\PY{n}{tS}\PY{p}{)}

\PY{k}{for} \PY{n}{patt} \PY{o+ow}{in} \PY{n}{patterns}\PY{p}{:}
  \PY{n}{pos} \PY{o}{=} \PY{n}{np}\PY{o}{.}\PY{n}{where}\PY{p}{(}\PY{n}{patt} \PY{o}{==} \PY{n}{tS}\PY{p}{)}\PY{p}{[}\PY{l+m+mi}{0}\PY{p}{]}
  \PY{k}{if} \PY{n}{patt} \PY{o}{==} \PY{o}{\PYZhy{}}\PY{l+m+mi}{1}\PY{p}{:}
    \PY{n}{plt}\PY{o}{.}\PY{n}{scatter}\PY{p}{(}\PY{n}{x}\PY{p}{[}\PY{n}{pos}\PY{p}{,}\PY{l+m+mi}{0}\PY{p}{]}\PY{p}{,}\PY{n}{x}\PY{p}{[}\PY{n}{pos}\PY{p}{,}\PY{l+m+mi}{1}\PY{p}{]}\PY{p}{,}\PY{n}{color} \PY{o}{=} \PY{l+s+s1}{\PYZsq{}}\PY{l+s+s1}{g}\PY{l+s+s1}{\PYZsq{}}\PY{p}{,} \PY{n}{s} \PY{o}{=}\PY{l+m+mi}{200}\PY{p}{,} \PY{n}{marker} \PY{o}{=} \PY{l+s+s1}{\PYZsq{}}\PY{l+s+s1}{o}\PY{l+s+s1}{\PYZsq{}}\PY{p}{,} \PY{n}{alpha} \PY{o}{=} \PY{l+m+mf}{0.8}\PY{p}{)}
  \PY{k}{else}\PY{p}{:}
    \PY{n}{plt}\PY{o}{.}\PY{n}{scatter}\PY{p}{(}\PY{n}{x}\PY{p}{[}\PY{n}{pos}\PY{p}{,}\PY{l+m+mi}{0}\PY{p}{]}\PY{p}{,}\PY{n}{x}\PY{p}{[}\PY{n}{pos}\PY{p}{,}\PY{l+m+mi}{1}\PY{p}{]}\PY{p}{,} \PY{n}{color}\PY{o}{=} \PY{l+s+s1}{\PYZsq{}}\PY{l+s+s1}{b}\PY{l+s+s1}{\PYZsq{}}\PY{p}{,} \PY{n}{s} \PY{o}{=}\PY{l+m+mi}{200}\PY{p}{,} \PY{n}{marker} \PY{o}{=} \PY{l+s+s1}{\PYZsq{}}\PY{l+s+s1}{x}\PY{l+s+s1}{\PYZsq{}}\PY{p}{,} \PY{n}{alpha} \PY{o}{=} \PY{l+m+mf}{0.8}\PY{p}{)}

\PY{n}{x} \PY{o}{=} \PY{n}{np}\PY{o}{.}\PY{n}{linspace}\PY{p}{(}\PY{o}{\PYZhy{}}\PY{l+m+mi}{1}\PY{p}{,}\PY{l+m+mi}{15}\PY{p}{)}

\PY{n}{yS} \PY{o}{=} \PY{n}{w\PYZus{}iS}\PY{p}{[}\PY{l+m+mi}{2}\PY{p}{]} \PY{o}{/} \PY{n}{w\PYZus{}iS}\PY{p}{[}\PY{l+m+mi}{1}\PY{p}{]} \PY{o}{\PYZhy{}} \PY{p}{(}\PY{n}{x} \PY{o}{*} \PY{n}{w\PYZus{}iS}\PY{p}{[}\PY{l+m+mi}{0}\PY{p}{]}\PY{p}{)} \PY{o}{/} \PY{n}{w\PYZus{}iS}\PY{p}{[}\PY{l+m+mi}{1}\PY{p}{]}

\PY{n}{plt}\PY{o}{.}\PY{n}{plot}\PY{p}{(}\PY{n}{x}\PY{p}{,}\PY{n}{yS}\PY{p}{,}\PY{l+s+s1}{\PYZsq{}}\PY{l+s+s1}{r}\PY{l+s+s1}{\PYZsq{}}\PY{p}{,} \PY{n}{linewidth} \PY{o}{=} \PY{l+m+mi}{2}\PY{p}{)}
\PY{n}{plt}\PY{o}{.}\PY{n}{title}\PY{p}{(}\PY{l+s+s1}{\PYZsq{}}\PY{l+s+s1}{Fronteras de decisión: Sigmoide}\PY{l+s+s1}{\PYZsq{}}\PY{p}{)}
\PY{n}{plt}\PY{o}{.}\PY{n}{show}\PY{p}{(}\PY{n}{fig}\PY{p}{)}
\end{Verbatim}
\end{tcolorbox}

    \begin{center}
    \adjustimage{max size={0.9\linewidth}{0.9\paperheight}}{sections/sigmoide/output_5_0.png}
    \end{center}
    { \hspace*{\fill} \\}
    
    \begin{tcolorbox}[breakable, size=fbox, boxrule=1pt, pad at break*=1mm,colback=cellbackground, colframe=cellborder]
\prompt{In}{incolor}{7}{\boxspacing}
\begin{Verbatim}[commandchars=\\\{\}]
\PY{n+nb}{print}\PY{p}{(}\PY{l+s+s1}{\PYZsq{}}\PY{l+s+s1}{\PYZsq{}}\PY{p}{)}
\PY{n+nb}{print}\PY{p}{(}\PY{l+s+s1}{\PYZsq{}}\PY{l+s+s1}{Pesos finales: }\PY{l+s+s1}{\PYZsq{}}\PY{p}{)}
\PY{k}{for} \PY{n}{i} \PY{o+ow}{in} \PY{n+nb}{range}\PY{p}{(}\PY{l+m+mi}{1}\PY{p}{)}\PY{p}{:}
    \PY{n}{res} \PY{o}{=} \PY{n+nb}{str}\PY{p}{(}\PY{n}{w\PYZus{}iS}\PY{p}{)}
    \PY{n+nb}{print} \PY{p}{(}\PY{n}{res}\PY{p}{)}
\PY{n+nb}{print}\PY{p}{(}\PY{l+s+s1}{\PYZsq{}}\PY{l+s+s1}{\PYZsq{}}\PY{p}{)}

\PY{c+c1}{\PYZsh{} Dot product}
\PY{n}{a} \PY{o}{=} \PY{n}{np}\PY{o}{.}\PY{n}{dot}\PY{p}{(}\PY{n}{xS}\PY{p}{,}\PY{n}{w\PYZus{}iS}\PY{p}{)}

\PY{c+c1}{\PYZsh{}Funcion sigmoide}
\PY{n}{y} \PY{o}{=} \PY{n}{sigmoid}\PY{p}{(}\PY{n}{a}\PY{p}{)}

\PY{n+nb}{print}\PY{p}{(}\PY{l+s+s1}{\PYZsq{}}\PY{l+s+s1}{Sigmoide}\PY{l+s+s1}{\PYZsq{}}\PY{p}{)}
\PY{n+nb}{print}\PY{p}{(}\PY{l+s+s1}{\PYZsq{}}\PY{l+s+s1}{Meta:    Predicción:}\PY{l+s+s1}{\PYZsq{}}\PY{p}{)}
\PY{k}{for} \PY{n}{i} \PY{o+ow}{in} \PY{n+nb}{range}\PY{p}{(}\PY{n+nb}{len}\PY{p}{(}\PY{n}{y}\PY{p}{)}\PY{p}{)}\PY{p}{:}
    \PY{n}{resS} \PY{o}{=} \PY{n+nb}{str}\PY{p}{(}\PY{n}{tS}\PY{p}{[}\PY{n}{i}\PY{p}{]}\PY{p}{)} \PY{o}{+} \PY{l+s+s1}{\PYZsq{}}\PY{l+s+s1}{\PYZhy{}\PYZhy{}\PYZhy{}\PYZhy{}\PYZhy{}\PYZhy{}\PYZhy{}\PYZhy{}}\PY{l+s+s1}{\PYZsq{}} \PY{o}{+} \PY{n+nb}{str}\PY{p}{(}\PY{n}{y}\PY{p}{[}\PY{n}{i}\PY{p}{]}\PY{p}{)}
    \PY{n+nb}{print}\PY{p}{(}\PY{n}{resS}\PY{p}{)}

\PY{n+nb}{print}\PY{p}{(}\PY{l+s+sa}{f}\PY{l+s+s1}{\PYZsq{}}\PY{l+s+se}{\PYZbs{}n}\PY{l+s+s1}{Tiempo requerido sigmoide: }\PY{l+s+si}{\PYZob{}}\PY{n}{tocS} \PY{o}{\PYZhy{}} \PY{n}{ticS}\PY{l+s+si}{:}\PY{l+s+s1}{.5f}\PY{l+s+si}{\PYZcb{}}\PY{l+s+s1}{ ms.}\PY{l+s+s1}{\PYZsq{}}\PY{p}{)}
\PY{n+nb}{print}\PY{p}{(}\PY{l+s+sa}{f}\PY{l+s+s1}{\PYZsq{}}\PY{l+s+se}{\PYZbs{}n}\PY{l+s+s1}{Épocas requeridas: }\PY{l+s+si}{\PYZob{}}\PY{n}{epochS}\PY{l+s+si}{\PYZcb{}}\PY{l+s+s1}{.}\PY{l+s+se}{\PYZbs{}n}\PY{l+s+s1}{\PYZsq{}}\PY{p}{)}
\end{Verbatim}
\end{tcolorbox}

    \begin{Verbatim}[commandchars=\\\{\}]

Pesos finales:
[[ -1.04760182]
 [ -1.70487723]
 [-15.01514102]]

Sigmoide
Meta:    Predicción:
[1]--------[0.93647409]
[1]--------[0.98210584]
[1]--------[0.97378601]
[-1]--------[0.00197731]
[-1]--------[8.55055335e-05]
[-1]--------[0.00031827]

Tiempo requerido sigmoide: 0.48261 ms.

Épocas requeridas: 3000.

    \end{Verbatim}

    \begin{tcolorbox}[breakable, size=fbox, boxrule=1pt, pad at break*=1mm,colback=cellbackground, colframe=cellborder]
\prompt{In}{incolor}{8}{\boxspacing}
\begin{Verbatim}[commandchars=\\\{\}]
\PY{c+c1}{\PYZsh{}Prueba}
\PY{c+c1}{\PYZsh{} Plotting Decision Boundaries}
\PY{n}{plt}\PY{o}{.}\PY{n}{xlim}\PY{p}{(}\PY{p}{[}\PY{o}{\PYZhy{}}\PY{l+m+mf}{1.0}\PY{p}{,} \PY{l+m+mf}{15.0}\PY{p}{]}\PY{p}{)}
\PY{n}{plt}\PY{o}{.}\PY{n}{ylim}\PY{p}{(}\PY{p}{[}\PY{o}{\PYZhy{}}\PY{l+m+mf}{1.0}\PY{p}{,} \PY{l+m+mf}{15.0}\PY{p}{]}\PY{p}{)}


\PY{n}{plt}\PY{o}{.}\PY{n}{scatter}\PY{p}{(}\PY{l+m+mi}{5}\PY{p}{,} \PY{l+m+mi}{5}\PY{p}{,} \PY{n}{color} \PY{o}{=} \PY{l+s+s1}{\PYZsq{}}\PY{l+s+s1}{g}\PY{l+s+s1}{\PYZsq{}}\PY{p}{,} \PY{n}{s} \PY{o}{=} \PY{l+m+mi}{200}\PY{p}{,} \PY{n}{marker} \PY{o}{=} \PY{l+s+s1}{\PYZsq{}}\PY{l+s+s1}{o}\PY{l+s+s1}{\PYZsq{}}\PY{p}{,} \PY{n}{alpha} \PY{o}{=} \PY{l+m+mf}{0.8}\PY{p}{)}
\PY{n}{plt}\PY{o}{.}\PY{n}{scatter}\PY{p}{(}\PY{l+m+mi}{6}\PY{p}{,} \PY{l+m+mi}{8}\PY{p}{,} \PY{n}{color} \PY{o}{=} \PY{l+s+s1}{\PYZsq{}}\PY{l+s+s1}{b}\PY{l+s+s1}{\PYZsq{}}\PY{p}{,} \PY{n}{s} \PY{o}{=} \PY{l+m+mi}{200}\PY{p}{,} \PY{n}{marker} \PY{o}{=} \PY{l+s+s1}{\PYZsq{}}\PY{l+s+s1}{x}\PY{l+s+s1}{\PYZsq{}}\PY{p}{,} \PY{n}{alpha} \PY{o}{=} \PY{l+m+mf}{0.8}\PY{p}{)}

\PY{n}{x1} \PY{o}{=} \PY{n}{np}\PY{o}{.}\PY{n}{linspace}\PY{p}{(}\PY{o}{\PYZhy{}}\PY{l+m+mi}{1}\PY{p}{,} \PY{l+m+mi}{15}\PY{p}{)}
\PY{n}{x2} \PY{o}{=} \PY{n}{w\PYZus{}iS}\PY{p}{[}\PY{l+m+mi}{2}\PY{p}{]} \PY{o}{/} \PY{n}{w\PYZus{}iS}\PY{p}{[}\PY{l+m+mi}{1}\PY{p}{]} \PY{o}{\PYZhy{}} \PY{p}{(}\PY{n}{x1} \PY{o}{*} \PY{n}{w\PYZus{}iS}\PY{p}{[}\PY{l+m+mi}{0}\PY{p}{]}\PY{p}{)} \PY{o}{/} \PY{n}{w\PYZus{}iS}\PY{p}{[}\PY{l+m+mi}{1}\PY{p}{]}

\PY{n}{plt}\PY{o}{.}\PY{n}{plot}\PY{p}{(}\PY{n}{x1}\PY{p}{,} \PY{n}{x2}\PY{p}{,} \PY{l+s+s1}{\PYZsq{}}\PY{l+s+s1}{red}\PY{l+s+s1}{\PYZsq{}}\PY{p}{,} \PY{n}{linewidth} \PY{o}{=} \PY{l+m+mi}{2}\PY{p}{)}
\PY{n}{plt}\PY{o}{.}\PY{n}{title}\PY{p}{(}\PY{l+s+s1}{\PYZsq{}}\PY{l+s+s1}{Prueba del Sigmoide}\PY{l+s+s1}{\PYZsq{}}\PY{p}{)}
\PY{n}{plt}\PY{o}{.}\PY{n}{show}\PY{p}{(}\PY{p}{)}
\end{Verbatim}
\end{tcolorbox}

    \begin{center}
    \adjustimage{max size={0.9\linewidth}{0.9\paperheight}}{sections/sigmoide/output_7_0.png}
    \end{center}
    { \hspace*{\fill} \\}
    

\clearpage