\section{Perceptrón}
Dada la siguiente tabla con dos clases A y B:

% Please add the following required packages to your document preamble:
% \usepackage[table,xcdraw]{xcolor}
% If you use beamer only pass "xcolor=table" option, i.e. \documentclass[xcolor=table]{beamer}
\begin{table}[!htb]
    \centering
    \begin{tabular}{|
    >{\columncolor[HTML]{3166FF}}l |l|l|l|l|l|l|l|l|l|l|l|l|}
    \hline
    {\color[HTML]{FFFFFF} 12} &  &  &  &  &  &  &  &  &  &  &  &  \\ \hline
    {\color[HTML]{FFFFFF} 11} &  &  &  &  &  &  &  &  &  &  &  &  \\ \hline
    {\color[HTML]{FFFFFF} 10} &  &  &  & \cellcolor[HTML]{FE0000}B &  &  & \cellcolor[HTML]{FE0000}B &  &  &  &  &  \\ \hline
    {\color[HTML]{FFFFFF} 9} &  &  &  &  &  &  &  &  &  &  &  &  \\ \hline
    {\color[HTML]{FFFFFF} 8} &  &  &  &  &  &  &  &  & \cellcolor[HTML]{FE0000}B &  &  &  \\ \hline
    {\color[HTML]{FFFFFF} 7} &  &  &  &  &  &  &  &  &  &  &  &  \\ \hline
    {\color[HTML]{FFFFFF} 6} &  & \cellcolor[HTML]{34FF34}A &  &  &  &  &  &  &  &  &  &  \\ \hline
    {\color[HTML]{FFFFFF} 5} &  &  &  &  &  &  &  &  &  &  &  &  \\ \hline
    {\color[HTML]{FFFFFF} 4} &  &  &  & \cellcolor[HTML]{34FF34}A &  &  &  &  &  &  &  &  \\ \hline
    {\color[HTML]{FFFFFF} 3} &  &  &  &  &  & \cellcolor[HTML]{34FF34}A &  &  &  &  &  &  \\ \hline
    {\color[HTML]{FFFFFF} 2} &  &  &  &  &  &  &  &  &  &  &  &  \\ \hline
    {\color[HTML]{FFFFFF} 1} &  &  &  &  &  &  &  &  &  &  &  &  \\ \hline
    {\color[HTML]{FFFFFF} 0} & \cellcolor[HTML]{3166FF}{\color[HTML]{FFFFFF} 1} & \cellcolor[HTML]{3166FF}{\color[HTML]{FFFFFF} 2} & \cellcolor[HTML]{3166FF}{\color[HTML]{FFFFFF} 3} & \cellcolor[HTML]{3166FF}{\color[HTML]{FFFFFF} 4} & \cellcolor[HTML]{3166FF}{\color[HTML]{FFFFFF} 5} & \cellcolor[HTML]{3166FF}{\color[HTML]{FFFFFF} 6} & \cellcolor[HTML]{3166FF}{\color[HTML]{FFFFFF} 7} & \cellcolor[HTML]{3166FF}{\color[HTML]{FFFFFF} 8} & \cellcolor[HTML]{3166FF}{\color[HTML]{FFFFFF} 9} & \cellcolor[HTML]{3166FF}{\color[HTML]{FFFFFF} 10} & \cellcolor[HTML]{3166FF}{\color[HTML]{FFFFFF} 11} & \cellcolor[HTML]{3166FF}{\color[HTML]{FFFFFF} 12} \\ \hline
    \end{tabular}
\end{table}

\begin{itemize}
    \item Entrene un perceptron con función de activación limitadora y con bias mediante la regla del perceptrón. Proponga una vector de pesos iniciales $W_0=(w_1,w_2,w_3 )^T$ y un $\alpha$. Muestre la línea inicial, el vector de pesos final, así como la línea de separación final.
	\item Use el perceptrón entrenado para determinar la clase A o B de los puntos con coordenadas (5,5) y (6,8).
\end{itemize}

Tomé como base los conocimientos adquiridos en la tarea 1 para proceder con el código, con ayuda de
lo desarrollado en las clases, no tuve complicaciones al editar las líneas necesarias para resolver el 
problema planteado.

\begin{tcolorbox}[breakable, size=fbox, boxrule=1pt, pad at break*=1mm,colback=cellbackground, colframe=cellborder]
\prompt{In}{incolor}{1}{\boxspacing}
\begin{Verbatim}[commandchars=\\\{\}]
\PY{c+c1}{\PYZsh{} Importando librerías}

\PY{k+kn}{import} \PY{n+nn}{numpy} \PY{k}{as} \PY{n+nn}{np}
\PY{k+kn}{import} \PY{n+nn}{matplotlib}\PY{n+nn}{.}\PY{n+nn}{pyplot} \PY{k}{as} \PY{n+nn}{plt}
\PY{k+kn}{import} \PY{n+nn}{time}
\PY{n}{err\PYZus{}vector} \PY{o}{=} \PY{p}{[}\PY{p}{]} \PY{c+c1}{\PYZsh{}vector de error}
\PY{n}{w\PYZus{}f} \PY{o}{=} \PY{p}{[}\PY{p}{]} \PY{c+c1}{\PYZsh{}vector de pesos finales}
\PY{c+c1}{\PYZsh{}Entradas}
\PY{n}{x} \PY{o}{=} \PY{n}{np}\PY{o}{.}\PY{n}{array}\PY{p}{(}\PY{p}{[}
    \PY{p}{[}\PY{l+m+mi}{2}\PY{p}{,}\PY{l+m+mi}{6}\PY{p}{]}\PY{p}{,}
    \PY{p}{[}\PY{l+m+mi}{4}\PY{p}{,}\PY{l+m+mi}{4}\PY{p}{]}\PY{p}{,}
    \PY{p}{[}\PY{l+m+mi}{6}\PY{p}{,}\PY{l+m+mi}{3}\PY{p}{]}\PY{p}{,}
    \PY{p}{[}\PY{l+m+mi}{4}\PY{p}{,}\PY{l+m+mi}{10}\PY{p}{]}\PY{p}{,}
    \PY{p}{[}\PY{l+m+mi}{7}\PY{p}{,}\PY{l+m+mi}{10}\PY{p}{]}\PY{p}{,}
    \PY{p}{[}\PY{l+m+mi}{9}\PY{p}{,}\PY{l+m+mi}{8}\PY{p}{]}
\PY{p}{]}\PY{p}{)}
\PY{n+nb}{print}\PY{p}{(}\PY{l+s+sa}{f}\PY{l+s+s2}{\PYZdq{}}\PY{l+s+s2}{Entradas: }\PY{l+s+se}{\PYZbs{}n}\PY{l+s+si}{\PYZob{}}\PY{n}{x}\PY{l+s+si}{\PYZcb{}}\PY{l+s+s2}{\PYZdq{}}\PY{p}{)}
\end{Verbatim}
\end{tcolorbox}

    \begin{Verbatim}[commandchars=\\\{\}]
Entradas:
[[ 2  6]
 [ 4  4]
 [ 6  3]
 [ 4 10]
 [ 7 10]
 [ 9  8]]
    \end{Verbatim}

    \begin{tcolorbox}[breakable, size=fbox, boxrule=1pt, pad at break*=1mm,colback=cellbackground, colframe=cellborder]
\prompt{In}{incolor}{2}{\boxspacing}
\begin{Verbatim}[commandchars=\\\{\}]
\PY{c+c1}{\PYZsh{}Targets}
\PY{n}{t} \PY{o}{=} \PY{n}{np}\PY{o}{.}\PY{n}{array}\PY{p}{(}\PY{p}{[}
    \PY{p}{[}\PY{l+m+mi}{0}\PY{p}{]}\PY{p}{,}\PY{p}{[}\PY{l+m+mi}{0}\PY{p}{]}\PY{p}{,}\PY{p}{[}\PY{l+m+mi}{0}\PY{p}{]}\PY{p}{,}\PY{p}{[}\PY{l+m+mi}{1}\PY{p}{]}\PY{p}{,}\PY{p}{[}\PY{l+m+mi}{1}\PY{p}{]}\PY{p}{,}\PY{p}{[}\PY{l+m+mi}{1}\PY{p}{]}\PY{p}{,}
\PY{p}{]}\PY{p}{)}
\PY{n+nb}{print}\PY{p}{(}\PY{l+s+sa}{f}\PY{l+s+s2}{\PYZdq{}}\PY{l+s+s2}{Objetivos: }\PY{l+s+se}{\PYZbs{}n}\PY{l+s+si}{\PYZob{}}\PY{n}{t}\PY{l+s+si}{\PYZcb{}}\PY{l+s+s2}{\PYZdq{}}\PY{p}{)}
\end{Verbatim}
\end{tcolorbox}

    \begin{Verbatim}[commandchars=\\\{\}]
Objetivos:
[[0]
 [0]
 [0]
 [1]
 [1]
 [1]]
    \end{Verbatim}

    \begin{tcolorbox}[breakable, size=fbox, boxrule=1pt, pad at break*=1mm,colback=cellbackground, colframe=cellborder]
\prompt{In}{incolor}{3}{\boxspacing}
\begin{Verbatim}[commandchars=\\\{\}]
\PY{c+c1}{\PYZsh{}bias}
\PY{n}{bias} \PY{o}{=} \PY{n}{np}\PY{o}{.}\PY{n}{shape}\PY{p}{(}\PY{n}{x}\PY{p}{)}\PY{p}{[}\PY{l+m+mi}{0}\PY{p}{]} \PY{c+c1}{\PYZsh{}[0] \PYZhy{}\PYZgt{} columnas [1] \PYZhy{}\PYZgt{} filas}
\PY{n}{bias} \PY{o}{=} \PY{o}{\PYZhy{}}\PY{l+m+mi}{1}\PY{o}{*}\PY{n}{np}\PY{o}{.}\PY{n}{ones}\PY{p}{(}\PY{p}{(}\PY{n}{bias}\PY{p}{,} \PY{l+m+mi}{1}\PY{p}{)}\PY{p}{)}
\PY{n+nb}{print}\PY{p}{(}\PY{l+s+sa}{f}\PY{l+s+s2}{\PYZdq{}}\PY{l+s+s2}{bias:}\PY{l+s+se}{\PYZbs{}n}\PY{l+s+s2}{ }\PY{l+s+si}{\PYZob{}}\PY{n}{bias}\PY{l+s+si}{\PYZcb{}}\PY{l+s+s2}{\PYZdq{}}\PY{p}{)}
\end{Verbatim}
\end{tcolorbox}

    \begin{Verbatim}[commandchars=\\\{\}]
bias:
 [[-1.]
 [-1.]
 [-1.]
 [-1.]
 [-1.]
 [-1.]]
    \end{Verbatim}

    \begin{tcolorbox}[breakable, size=fbox, boxrule=1pt, pad at break*=1mm,colback=cellbackground, colframe=cellborder]
\prompt{In}{incolor}{4}{\boxspacing}
\begin{Verbatim}[commandchars=\\\{\}]
\PY{c+c1}{\PYZsh{}Vector aumentado}
\PY{c+c1}{\PYZsh{}Se tiene que usar el bias}
\PY{n}{x} \PY{o}{=} \PY{n}{np}\PY{o}{.}\PY{n}{concatenate}\PY{p}{(}\PY{p}{[}\PY{n}{x}\PY{p}{,} \PY{n}{bias}\PY{p}{]}\PY{p}{,} \PY{n}{axis} \PY{o}{=} \PY{l+m+mi}{1}\PY{p}{)}
\PY{n+nb}{print}\PY{p}{(}\PY{l+s+sa}{f}\PY{l+s+s2}{\PYZdq{}}\PY{l+s+s2}{Vector aumentado: }\PY{l+s+se}{\PYZbs{}n}\PY{l+s+s2}{ }\PY{l+s+si}{\PYZob{}}\PY{n}{x}\PY{l+s+si}{\PYZcb{}}\PY{l+s+s2}{\PYZdq{}}\PY{p}{)}
\end{Verbatim}
\end{tcolorbox}

    \begin{Verbatim}[commandchars=\\\{\}]
Vector aumentado:
 [[ 2.  6. -1.]
 [ 4.  4. -1.]
 [ 6.  3. -1.]
 [ 4. 10. -1.]
 [ 7. 10. -1.]
 [ 9.  8. -1.]]
    \end{Verbatim}

    \begin{tcolorbox}[breakable, size=fbox, boxrule=1pt, pad at break*=1mm,colback=cellbackground, colframe=cellborder]
\prompt{In}{incolor}{5}{\boxspacing}
\begin{Verbatim}[commandchars=\\\{\}]
\PY{c+c1}{\PYZsh{}Pesos}
\PY{n}{w\PYZus{}i} \PY{o}{=} \PY{n}{np}\PY{o}{.}\PY{n}{array}\PY{p}{(}\PY{p}{[}\PY{p}{[}\PY{l+m+mf}{0.25}\PY{p}{]}\PY{p}{,}\PY{p}{[}\PY{l+m+mf}{0.25}\PY{p}{]}\PY{p}{,} \PY{p}{[}\PY{l+m+mf}{1.0}\PY{p}{]}\PY{p}{]}\PY{p}{)}
\PY{n+nb}{print}\PY{p}{(}\PY{l+s+sa}{f}\PY{l+s+s2}{\PYZdq{}}\PY{l+s+s2}{Pesos:}\PY{l+s+se}{\PYZbs{}n}\PY{l+s+s2}{ }\PY{l+s+si}{\PYZob{}}\PY{n}{w\PYZus{}i}\PY{l+s+si}{\PYZcb{}}\PY{l+s+s2}{\PYZdq{}}\PY{p}{)}
\end{Verbatim}
\end{tcolorbox}

    \begin{Verbatim}[commandchars=\\\{\}]
Pesos:
 [[0.25]
 [0.25]
 [1.  ]]
    \end{Verbatim}

    \begin{tcolorbox}[breakable, size=fbox, boxrule=1pt, pad at break*=1mm,colback=cellbackground, colframe=cellborder]
\prompt{In}{incolor}{6}{\boxspacing}
\begin{Verbatim}[commandchars=\\\{\}]
\PY{n}{alpha} \PY{o}{=} \PY{l+m+mf}{0.25}
\PY{c+c1}{\PYZsh{}Tiempo}
\PY{n}{tic} \PY{o}{=}  \PY{n}{time}\PY{o}{.}\PY{n}{time}\PY{p}{(}\PY{p}{)} \PY{c+c1}{\PYZsh{}Se inicia cronometro}
\end{Verbatim}
\end{tcolorbox}

    \begin{tcolorbox}[breakable, size=fbox, boxrule=1pt, pad at break*=1mm,colback=cellbackground, colframe=cellborder]
\prompt{In}{incolor}{7}{\boxspacing}
\begin{Verbatim}[commandchars=\\\{\}]
\PY{c+c1}{\PYZsh{}Regla del perceptron}
\PY{n}{a} \PY{o}{=} \PY{n}{np}\PY{o}{.}\PY{n}{dot}\PY{p}{(}\PY{n}{x}\PY{p}{,} \PY{n}{w\PYZus{}i}\PY{p}{)}
\PY{n+nb}{print}\PY{p}{(}\PY{l+s+sa}{f}\PY{l+s+s2}{\PYZdq{}}\PY{l+s+s2}{a: }\PY{l+s+si}{\PYZob{}}\PY{n}{a}\PY{l+s+si}{\PYZcb{}}\PY{l+s+s2}{\PYZdq{}}\PY{p}{)}
\end{Verbatim}
\end{tcolorbox}

    \begin{Verbatim}[commandchars=\\\{\}]
a: [[1.  ]
 [1.  ]
 [1.25]
 [2.5 ]
 [3.25]
 [3.25]]
    \end{Verbatim}

    \begin{tcolorbox}[breakable, size=fbox, boxrule=1pt, pad at break*=1mm,colback=cellbackground, colframe=cellborder]
\prompt{In}{incolor}{8}{\boxspacing}
\begin{Verbatim}[commandchars=\\\{\}]
\PY{c+c1}{\PYZsh{}Funciòn limite duro}
\PY{n}{y} \PY{o}{=} \PY{n}{np}\PY{o}{.}\PY{n}{uint32}\PY{p}{(}\PY{n}{a} \PY{o}{\PYZgt{}}\PY{o}{=} \PY{l+m+mi}{0}\PY{p}{)}
\PY{n+nb}{print}\PY{p}{(}\PY{l+s+sa}{f}\PY{l+s+s2}{\PYZdq{}}\PY{l+s+s2}{y:}\PY{l+s+se}{\PYZbs{}n}\PY{l+s+s2}{ }\PY{l+s+si}{\PYZob{}}\PY{n}{y}\PY{l+s+si}{\PYZcb{}}\PY{l+s+s2}{\PYZdq{}}\PY{p}{)}
\end{Verbatim}
\end{tcolorbox}

    \begin{Verbatim}[commandchars=\\\{\}]
y:
 [[1]
 [1]
 [1]
 [1]
 [1]
 [1]]
    \end{Verbatim}

    \begin{tcolorbox}[breakable, size=fbox, boxrule=1pt, pad at break*=1mm,colback=cellbackground, colframe=cellborder]
\prompt{In}{incolor}{9}{\boxspacing}
\begin{Verbatim}[commandchars=\\\{\}]
\PY{c+c1}{\PYZsh{} Función de costo \PYZhy{}\PYZhy{} MSE Error cuadrático medio}
\PY{n}{err} \PY{o}{=} \PY{n+nb}{sum}\PY{p}{(}\PY{p}{(}\PY{n}{t} \PY{o}{\PYZhy{}} \PY{n}{y}\PY{p}{)} \PY{o}{*}\PY{o}{*} \PY{l+m+mi}{2}\PY{p}{)} \PY{o}{/} \PY{n+nb}{len}\PY{p}{(}\PY{n}{y}\PY{p}{)}
\PY{n}{err\PYZus{}vector}\PY{o}{.}\PY{n}{append}\PY{p}{(}\PY{n}{err}\PY{p}{)}
\PY{n+nb}{print}\PY{p}{(}\PY{l+s+sa}{f}\PY{l+s+s2}{\PYZdq{}}\PY{l+s+s2}{Error:}\PY{l+s+se}{\PYZbs{}n}\PY{l+s+si}{\PYZob{}}\PY{n}{err\PYZus{}vector}\PY{l+s+si}{\PYZcb{}}\PY{l+s+se}{\PYZbs{}n}\PY{l+s+s2}{\PYZdq{}}\PY{p}{)}
\PY{n}{epoch} \PY{o}{=} \PY{l+m+mi}{0}
\end{Verbatim}
\end{tcolorbox}

    \begin{Verbatim}[commandchars=\\\{\}]
Error:
[array([0.5])]

    \end{Verbatim}

    \begin{tcolorbox}[breakable, size=fbox, boxrule=1pt, pad at break*=1mm,colback=cellbackground, colframe=cellborder]
\prompt{In}{incolor}{10}{\boxspacing}
\begin{Verbatim}[commandchars=\\\{\}]
\PY{k}{while}\PY{p}{(}\PY{n+nb}{sum}\PY{p}{(}\PY{n}{y} \PY{o}{!=} \PY{n}{t}\PY{p}{)}\PY{p}{)}\PY{p}{:} \PY{c+c1}{\PYZsh{}mientras la salida de la RNA no sea igual al target}
  \PY{n}{epoch} \PY{o}{+}\PY{o}{=} \PY{l+m+mi}{1}
  \PY{c+c1}{\PYZsh{}print(f\PYZdq{}Epoca: \PYZob{}epoch\PYZcb{}\PYZbs{}n\PYZdq{})}

  \PY{k}{for} \PY{n}{i} \PY{o+ow}{in} \PY{n+nb}{range}\PY{p}{(}\PY{n}{np}\PY{o}{.}\PY{n}{shape}\PY{p}{(}\PY{n}{x}\PY{p}{)}\PY{p}{[}\PY{l+m+mi}{0}\PY{p}{]}\PY{p}{)}\PY{p}{:} \PY{c+c1}{\PYZsh{}Repetir segun la cantidad de filas de x}
    \PY{c+c1}{\PYZsh{} Producto punto}
    \PY{n}{a} \PY{o}{=} \PY{n}{np}\PY{o}{.}\PY{n}{dot}\PY{p}{(}\PY{n}{x}\PY{p}{[}\PY{n}{i}\PY{p}{]}\PY{p}{,} \PY{n}{w\PYZus{}i}\PY{p}{)}

    \PY{c+c1}{\PYZsh{}Limite Duro}
    \PY{n}{y}\PY{p}{[}\PY{n}{i}\PY{p}{]}\PY{o}{=} \PY{n}{np}\PY{o}{.}\PY{n}{uint32}\PY{p}{(}\PY{n}{a} \PY{o}{\PYZgt{}}\PY{o}{=} \PY{l+m+mi}{0}\PY{p}{)}

    \PY{c+c1}{\PYZsh{} Actualización de pesos}
    \PY{n}{x\PYZus{}p} \PY{o}{=} \PY{n}{np}\PY{o}{.}\PY{n}{reshape}\PY{p}{(}\PY{n}{x}\PY{p}{[}\PY{n}{i}\PY{p}{]}\PY{p}{,} \PY{p}{(}\PY{n+nb}{len}\PY{p}{(}\PY{n}{w\PYZus{}i}\PY{p}{)}\PY{p}{,} \PY{l+m+mi}{1}\PY{p}{)}\PY{p}{)} \PY{c+c1}{\PYZsh{}REACOMODA np.reshape(matriz,(filas,columnas))}
    \PY{n}{w\PYZus{}n} \PY{o}{=} \PY{n}{w\PYZus{}i} \PY{o}{+} \PY{n}{alpha} \PY{o}{*} \PY{p}{(}\PY{n}{t}\PY{p}{[}\PY{n}{i}\PY{p}{]} \PY{o}{\PYZhy{}} \PY{n}{y}\PY{p}{[}\PY{n}{i}\PY{p}{]}\PY{p}{)} \PY{o}{*} \PY{n}{x\PYZus{}p}
    \PY{n}{w\PYZus{}i} \PY{o}{=} \PY{n}{w\PYZus{}n}
    \PY{c+c1}{\PYZsh{}print(f\PYZdq{}Peso nuevo:\PYZbs{}n\PYZob{}w\PYZus{}i\PYZcb{}\PYZbs{}n\PYZdq{})}
 
  \PY{c+c1}{\PYZsh{} Función de costo \PYZhy{}\PYZhy{} MSE Error cuadrático medio}
  \PY{n}{err} \PY{o}{=} \PY{n+nb}{sum}\PY{p}{(}\PY{p}{(}\PY{n}{t} \PY{o}{\PYZhy{}} \PY{n}{y}\PY{p}{)} \PY{o}{*}\PY{o}{*} \PY{l+m+mi}{2}\PY{p}{)} \PY{o}{/} \PY{n+nb}{len}\PY{p}{(}\PY{n}{y}\PY{p}{)}
  \PY{n}{err\PYZus{}vector}\PY{o}{.}\PY{n}{append}\PY{p}{(}\PY{n}{err}\PY{p}{)}
  \PY{c+c1}{\PYZsh{}print(f\PYZdq{}Error:\PYZbs{}n\PYZob{}err\PYZcb{}\PYZbs{}n\PYZdq{})}

  \PY{c+c1}{\PYZsh{}Se añade al vector de pesos}
  \PY{n}{w\PYZus{}f}\PY{o}{.}\PY{n}{append}\PY{p}{(}\PY{n}{w\PYZus{}i}\PY{p}{)}
  \PY{c+c1}{\PYZsh{}print(f\PYZdq{}Vector de Pesos Finales:\PYZbs{}n\PYZob{}w\PYZus{}f\PYZcb{}\PYZbs{}n\PYZdq{})}


\PY{n}{toc} \PY{o}{=} \PY{n}{time}\PY{o}{.}\PY{n}{time}\PY{p}{(}\PY{p}{)} \PY{c+c1}{\PYZsh{}Paro cronometro}
\end{Verbatim}
\end{tcolorbox}

    \begin{tcolorbox}[breakable, size=fbox, boxrule=1pt, pad at break*=1mm,colback=cellbackground, colframe=cellborder]
\prompt{In}{incolor}{11}{\boxspacing}
\begin{Verbatim}[commandchars=\\\{\}]
\PY{c+c1}{\PYZsh{} Plotting Error \PYZhy{}\PYZhy{} Graph}
\PY{n}{plt}\PY{o}{.}\PY{n}{figure}\PY{p}{(}\PY{l+m+mi}{0}\PY{p}{)}
\PY{n}{plt}\PY{o}{.}\PY{n}{plot}\PY{p}{(}\PY{n}{err\PYZus{}vector}\PY{p}{,} \PY{n}{linewidth} \PY{o}{=} \PY{l+m+mi}{2}\PY{p}{)}
\PY{n}{plt}\PY{o}{.}\PY{n}{title}\PY{p}{(}\PY{l+s+s1}{\PYZsq{}}\PY{l+s+s1}{Gráfica de error: REGLA PERCEPTRON}\PY{l+s+s1}{\PYZsq{}}\PY{p}{)}
\PY{n}{plt}\PY{o}{.}\PY{n}{ylabel}\PY{p}{(}\PY{l+s+s1}{\PYZsq{}}\PY{l+s+s1}{Magnitud de Error}\PY{l+s+s1}{\PYZsq{}}\PY{p}{)}
\PY{n}{plt}\PY{o}{.}\PY{n}{xlabel}\PY{p}{(}\PY{l+s+s1}{\PYZsq{}}\PY{l+s+s1}{Épocas}\PY{l+s+s1}{\PYZsq{}}\PY{p}{)}
\PY{n}{plt}\PY{o}{.}\PY{n}{scatter}\PY{p}{(}\PY{n+nb}{len}\PY{p}{(}\PY{n}{err\PYZus{}vector}\PY{p}{)} \PY{o}{\PYZhy{}} \PY{l+m+mi}{1}\PY{p}{,} \PY{l+m+mi}{0}\PY{p}{,} \PY{n}{color} \PY{o}{=} \PY{l+s+s1}{\PYZsq{}}\PY{l+s+s1}{r}\PY{l+s+s1}{\PYZsq{}}\PY{p}{,} \PY{n}{s} \PY{o}{=} \PY{l+m+mi}{200}\PY{p}{,} \PY{n}{marker} \PY{o}{=} \PY{l+s+s1}{\PYZsq{}}\PY{l+s+s1}{o}\PY{l+s+s1}{\PYZsq{}}\PY{p}{,} \PY{n}{alpha} \PY{o}{=} \PY{l+m+mf}{0.8}\PY{p}{)}
\PY{n}{plt}\PY{o}{.}\PY{n}{show}\PY{p}{(}\PY{p}{)}

\PY{c+c1}{\PYZsh{} Plotting Decision Boundaries}
\PY{n}{plt}\PY{o}{.}\PY{n}{xlim}\PY{p}{(}\PY{p}{[}\PY{o}{\PYZhy{}}\PY{l+m+mf}{1.0}\PY{p}{,} \PY{l+m+mf}{15.0}\PY{p}{]}\PY{p}{)}
\PY{n}{plt}\PY{o}{.}\PY{n}{ylim}\PY{p}{(}\PY{p}{[}\PY{o}{\PYZhy{}}\PY{l+m+mf}{1.0}\PY{p}{,} \PY{l+m+mf}{15.0}\PY{p}{]}\PY{p}{)}

\PY{n}{patterns} \PY{o}{=} \PY{n}{np}\PY{o}{.}\PY{n}{unique}\PY{p}{(}\PY{n}{t}\PY{p}{)} \PY{c+c1}{\PYZsh{}Encuentra los elementos únicos de la matriz t}

\PY{k}{for} \PY{n}{patt} \PY{o+ow}{in} \PY{n}{patterns}\PY{p}{:}
  \PY{n}{pos} \PY{o}{=} \PY{n}{np}\PY{o}{.}\PY{n}{where}\PY{p}{(}\PY{n}{patt} \PY{o}{==} \PY{n}{t}\PY{p}{)}\PY{p}{[}\PY{l+m+mi}{0}\PY{p}{]} \PY{c+c1}{\PYZsh{} np.where(TRUE)[0]}
  \PY{k}{if} \PY{n}{patt} \PY{o}{==} \PY{l+m+mi}{0}\PY{p}{:}
    \PY{n}{plt}\PY{o}{.}\PY{n}{scatter}\PY{p}{(}\PY{n}{x}\PY{p}{[}\PY{n}{pos}\PY{p}{,} \PY{l+m+mi}{0}\PY{p}{]}\PY{p}{,} \PY{n}{x}\PY{p}{[}\PY{n}{pos}\PY{p}{,} \PY{l+m+mi}{1}\PY{p}{]}\PY{p}{,} \PY{n}{color} \PY{o}{=} \PY{l+s+s1}{\PYZsq{}}\PY{l+s+s1}{g}\PY{l+s+s1}{\PYZsq{}}\PY{p}{,} \PY{n}{s} \PY{o}{=} \PY{l+m+mi}{200}\PY{p}{,} \PY{n}{marker} \PY{o}{=} \PY{l+s+s1}{\PYZsq{}}\PY{l+s+s1}{o}\PY{l+s+s1}{\PYZsq{}}\PY{p}{,} \PY{n}{alpha} \PY{o}{=} \PY{l+m+mf}{0.8}\PY{p}{)}
  \PY{k}{else}\PY{p}{:}
    \PY{n}{plt}\PY{o}{.}\PY{n}{scatter}\PY{p}{(}\PY{n}{x}\PY{p}{[}\PY{n}{pos}\PY{p}{,} \PY{l+m+mi}{0}\PY{p}{]}\PY{p}{,} \PY{n}{x}\PY{p}{[}\PY{n}{pos}\PY{p}{,} \PY{l+m+mi}{1}\PY{p}{]}\PY{p}{,} \PY{n}{color} \PY{o}{=} \PY{l+s+s1}{\PYZsq{}}\PY{l+s+s1}{b}\PY{l+s+s1}{\PYZsq{}}\PY{p}{,} \PY{n}{s} \PY{o}{=} \PY{l+m+mi}{200}\PY{p}{,} \PY{n}{marker} \PY{o}{=} \PY{l+s+s1}{\PYZsq{}}\PY{l+s+s1}{x}\PY{l+s+s1}{\PYZsq{}}\PY{p}{,} \PY{n}{alpha} \PY{o}{=} \PY{l+m+mf}{0.8}\PY{p}{)}

\PY{n}{x1} \PY{o}{=} \PY{n}{np}\PY{o}{.}\PY{n}{linspace}\PY{p}{(}\PY{o}{\PYZhy{}}\PY{l+m+mi}{1}\PY{p}{,} \PY{l+m+mi}{15}\PY{p}{)}
\PY{n}{x2} \PY{o}{=} \PY{n}{w\PYZus{}i}\PY{p}{[}\PY{l+m+mi}{2}\PY{p}{]} \PY{o}{/} \PY{n}{w\PYZus{}i}\PY{p}{[}\PY{l+m+mi}{1}\PY{p}{]} \PY{o}{\PYZhy{}} \PY{p}{(}\PY{n}{x1} \PY{o}{*} \PY{n}{w\PYZus{}i}\PY{p}{[}\PY{l+m+mi}{0}\PY{p}{]}\PY{p}{)} \PY{o}{/} \PY{n}{w\PYZus{}i}\PY{p}{[}\PY{l+m+mi}{1}\PY{p}{]}

\PY{n}{plt}\PY{o}{.}\PY{n}{figure}\PY{p}{(}\PY{l+m+mi}{1}\PY{p}{)}
\PY{n}{plt}\PY{o}{.}\PY{n}{plot}\PY{p}{(}\PY{n}{x1}\PY{p}{,} \PY{n}{x2}\PY{p}{,} \PY{l+s+s1}{\PYZsq{}}\PY{l+s+s1}{red}\PY{l+s+s1}{\PYZsq{}}\PY{p}{,} \PY{n}{linewidth} \PY{o}{=} \PY{l+m+mi}{2}\PY{p}{)}
\PY{n}{plt}\PY{o}{.}\PY{n}{title}\PY{p}{(}\PY{l+s+s1}{\PYZsq{}}\PY{l+s+s1}{Fronteras de decisión: REGLA DEL PERCEPTRON}\PY{l+s+s1}{\PYZsq{}}\PY{p}{)}
\PY{n}{plt}\PY{o}{.}\PY{n}{show}\PY{p}{(}\PY{p}{)}
\end{Verbatim}
\end{tcolorbox}

    \begin{center}
    \adjustimage{max size={0.9\linewidth}{0.9\paperheight}}{sections/perceptron/output_10_0.png}
    \end{center}
    { \hspace*{\fill} \\}
    
    \begin{center}
    \adjustimage{max size={0.9\linewidth}{0.9\paperheight}}{sections/perceptron/output_10_1.png}
    \end{center}
    { \hspace*{\fill} \\}
    
    \begin{tcolorbox}[breakable, size=fbox, boxrule=1pt, pad at break*=1mm,colback=cellbackground, colframe=cellborder]
\prompt{In}{incolor}{12}{\boxspacing}
\begin{Verbatim}[commandchars=\\\{\}]
\PY{c+c1}{\PYZsh{}Pesos Finales}
\PY{n+nb}{print}\PY{p}{(}\PY{l+s+s1}{\PYZsq{}}\PY{l+s+se}{\PYZbs{}n}\PY{l+s+s1}{Pesos finales: }\PY{l+s+s1}{\PYZsq{}}\PY{p}{)}
\PY{k}{for} \PY{n}{i} \PY{o+ow}{in} \PY{n+nb}{range}\PY{p}{(}\PY{l+m+mi}{1}\PY{p}{)}\PY{p}{:}
    \PY{n}{res} \PY{o}{=} \PY{n+nb}{str}\PY{p}{(}\PY{n}{w\PYZus{}i}\PY{p}{)}
    \PY{n+nb}{print}\PY{p}{(}\PY{n}{res}\PY{p}{)}
    \PY{n+nb}{print}\PY{p}{(}\PY{p}{)}
\PY{c+c1}{\PYZsh{}\PYZhy{}\PYZhy{}\PYZhy{}\PYZhy{}\PYZhy{}\PYZhy{}\PYZhy{}\PYZhy{}\PYZhy{}\PYZhy{}\PYZhy{}\PYZhy{}\PYZhy{}\PYZhy{}\PYZhy{}\PYZhy{}\PYZhy{}\PYZhy{}\PYZhy{}\PYZhy{}\PYZhy{}\PYZhy{}\PYZhy{}\PYZhy{}\PYZhy{}\PYZhy{}\PYZhy{}\PYZhy{}\PYZhy{}\PYZhy{}\PYZhy{}\PYZhy{}\PYZhy{}\PYZhy{}\PYZhy{}\PYZhy{}\PYZhy{}\PYZhy{}\PYZhy{}\PYZhy{}\PYZhy{}\PYZhy{}\PYZhy{}\PYZhy{}\PYZhy{}\PYZhy{}\PYZhy{}\PYZhy{}\PYZhy{}\PYZhy{}\PYZhy{}\PYZhy{}\PYZhy{}\PYZhy{}\PYZhy{}\PYZhy{}\PYZhy{}\PYZhy{}\PYZhy{}\PYZhy{}\PYZhy{}\PYZhy{}\PYZhy{}\PYZhy{}\PYZhy{}\PYZhy{}}
\PY{c+c1}{\PYZsh{} Displaying Results}
\PY{n}{a} \PY{o}{=} \PY{n}{np}\PY{o}{.}\PY{n}{dot}\PY{p}{(}\PY{n}{x}\PY{p}{,} \PY{n}{w\PYZus{}i}\PY{p}{)}
\PY{n}{y}\PY{o}{=} \PY{n}{np}\PY{o}{.}\PY{n}{uint32}\PY{p}{(}\PY{n}{a} \PY{o}{\PYZgt{}}\PY{o}{=} \PY{l+m+mi}{0}\PY{p}{)}

\PY{n+nb}{print}\PY{p}{(}\PY{l+s+s1}{\PYZsq{}}\PY{l+s+s1}{REGLA DEL PERCEPTRON}\PY{l+s+s1}{\PYZsq{}}\PY{p}{)}
\PY{n+nb}{print}\PY{p}{(}\PY{l+s+s1}{\PYZsq{}}\PY{l+s+s1}{Meta:    Predicción:}\PY{l+s+s1}{\PYZsq{}}\PY{p}{)}
\PY{k}{for} \PY{n}{i} \PY{o+ow}{in} \PY{n+nb}{range}\PY{p}{(}\PY{n+nb}{len}\PY{p}{(}\PY{n}{y}\PY{p}{)}\PY{p}{)}\PY{p}{:}
    \PY{n}{res} \PY{o}{=} \PY{n+nb}{str}\PY{p}{(}\PY{n}{t}\PY{p}{[}\PY{n}{i}\PY{p}{]}\PY{p}{)} \PY{o}{+} \PY{l+s+s1}{\PYZsq{}}\PY{l+s+s1}{\PYZhy{}\PYZhy{}\PYZhy{}\PYZhy{}\PYZhy{}\PYZhy{}\PYZhy{}\PYZhy{}}\PY{l+s+s1}{\PYZsq{}} \PY{o}{+} \PY{n+nb}{str}\PY{p}{(}\PY{n}{y}\PY{p}{[}\PY{n}{i}\PY{p}{]}\PY{p}{)}
    \PY{n+nb}{print}\PY{p}{(}\PY{n}{res}\PY{p}{)}

\PY{n+nb}{print}\PY{p}{(}\PY{l+s+sa}{f}\PY{l+s+s1}{\PYZsq{}}\PY{l+s+se}{\PYZbs{}n}\PY{l+s+s1}{Tiempo requerido: }\PY{l+s+si}{\PYZob{}}\PY{n}{toc} \PY{o}{\PYZhy{}} \PY{n}{tic}\PY{l+s+si}{:}\PY{l+s+s1}{.5f}\PY{l+s+si}{\PYZcb{}}\PY{l+s+s1}{ ms.}\PY{l+s+s1}{\PYZsq{}}\PY{p}{)}
\PY{n+nb}{print}\PY{p}{(}\PY{l+s+sa}{f}\PY{l+s+s1}{\PYZsq{}}\PY{l+s+se}{\PYZbs{}n}\PY{l+s+s1}{Épocas requeridas: }\PY{l+s+si}{\PYZob{}}\PY{n}{epoch}\PY{l+s+si}{\PYZcb{}}\PY{l+s+s1}{.}\PY{l+s+s1}{\PYZsq{}}\PY{p}{)}
\end{Verbatim}
\end{tcolorbox}

    \begin{Verbatim}[commandchars=\\\{\}]

Pesos finales:
[[0. ]
 [0.5]
 [3.5]]

REGLA DEL PERCEPTRON
Meta:    Predicción:
[0]--------[0]
[0]--------[0]
[0]--------[0]
[1]--------[1]
[1]--------[1]
[1]--------[1]

Tiempo requerido: 0.07300 ms.

Épocas requeridas: 14.
    \end{Verbatim}

    \begin{tcolorbox}[breakable, size=fbox, boxrule=1pt, pad at break*=1mm,colback=cellbackground, colframe=cellborder]
\prompt{In}{incolor}{13}{\boxspacing}
\begin{Verbatim}[commandchars=\\\{\}]
\PY{c+c1}{\PYZsh{}Prueba}
\PY{c+c1}{\PYZsh{} Plotting Decision Boundaries}
\PY{n}{plt}\PY{o}{.}\PY{n}{xlim}\PY{p}{(}\PY{p}{[}\PY{o}{\PYZhy{}}\PY{l+m+mf}{1.0}\PY{p}{,} \PY{l+m+mf}{15.0}\PY{p}{]}\PY{p}{)}
\PY{n}{plt}\PY{o}{.}\PY{n}{ylim}\PY{p}{(}\PY{p}{[}\PY{o}{\PYZhy{}}\PY{l+m+mf}{1.0}\PY{p}{,} \PY{l+m+mf}{15.0}\PY{p}{]}\PY{p}{)}


\PY{n}{plt}\PY{o}{.}\PY{n}{scatter}\PY{p}{(}\PY{l+m+mi}{5}\PY{p}{,} \PY{l+m+mi}{5}\PY{p}{,} \PY{n}{color} \PY{o}{=} \PY{l+s+s1}{\PYZsq{}}\PY{l+s+s1}{g}\PY{l+s+s1}{\PYZsq{}}\PY{p}{,} \PY{n}{s} \PY{o}{=} \PY{l+m+mi}{200}\PY{p}{,} \PY{n}{marker} \PY{o}{=} \PY{l+s+s1}{\PYZsq{}}\PY{l+s+s1}{o}\PY{l+s+s1}{\PYZsq{}}\PY{p}{,} \PY{n}{alpha} \PY{o}{=} \PY{l+m+mf}{0.8}\PY{p}{)}
\PY{n}{plt}\PY{o}{.}\PY{n}{scatter}\PY{p}{(}\PY{l+m+mi}{6}\PY{p}{,} \PY{l+m+mi}{8}\PY{p}{,} \PY{n}{color} \PY{o}{=} \PY{l+s+s1}{\PYZsq{}}\PY{l+s+s1}{b}\PY{l+s+s1}{\PYZsq{}}\PY{p}{,} \PY{n}{s} \PY{o}{=} \PY{l+m+mi}{200}\PY{p}{,} \PY{n}{marker} \PY{o}{=} \PY{l+s+s1}{\PYZsq{}}\PY{l+s+s1}{x}\PY{l+s+s1}{\PYZsq{}}\PY{p}{,} \PY{n}{alpha} \PY{o}{=} \PY{l+m+mf}{0.8}\PY{p}{)}

\PY{n}{x1} \PY{o}{=} \PY{n}{np}\PY{o}{.}\PY{n}{linspace}\PY{p}{(}\PY{o}{\PYZhy{}}\PY{l+m+mi}{1}\PY{p}{,} \PY{l+m+mi}{15}\PY{p}{)}
\PY{n}{x2} \PY{o}{=} \PY{n}{w\PYZus{}i}\PY{p}{[}\PY{l+m+mi}{2}\PY{p}{]} \PY{o}{/} \PY{n}{w\PYZus{}i}\PY{p}{[}\PY{l+m+mi}{1}\PY{p}{]} \PY{o}{\PYZhy{}} \PY{p}{(}\PY{n}{x1} \PY{o}{*} \PY{n}{w\PYZus{}i}\PY{p}{[}\PY{l+m+mi}{0}\PY{p}{]}\PY{p}{)} \PY{o}{/} \PY{n}{w\PYZus{}i}\PY{p}{[}\PY{l+m+mi}{1}\PY{p}{]}

\PY{n}{plt}\PY{o}{.}\PY{n}{plot}\PY{p}{(}\PY{n}{x1}\PY{p}{,} \PY{n}{x2}\PY{p}{,} \PY{l+s+s1}{\PYZsq{}}\PY{l+s+s1}{red}\PY{l+s+s1}{\PYZsq{}}\PY{p}{,} \PY{n}{linewidth} \PY{o}{=} \PY{l+m+mi}{2}\PY{p}{)}
\PY{n}{plt}\PY{o}{.}\PY{n}{title}\PY{p}{(}\PY{l+s+s1}{\PYZsq{}}\PY{l+s+s1}{Prueba del PERCEPTRÓN}\PY{l+s+s1}{\PYZsq{}}\PY{p}{)}
\PY{n}{plt}\PY{o}{.}\PY{n}{show}\PY{p}{(}\PY{p}{)}
\end{Verbatim}
\end{tcolorbox}

    \begin{center}
    \adjustimage{max size={0.9\linewidth}{0.9\paperheight}}{sections/perceptron/output_12_0.png}
    \end{center}
    { \hspace*{\fill} \\}
    

    % Add a bibliography block to the postdoc
    
    