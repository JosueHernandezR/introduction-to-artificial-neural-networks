\section{Adaline}
Dada la siguiente tabla con dos clases A y B:

% Please add the following required packages to your document preamble:
% \usepackage[table,xcdraw]{xcolor}
% If you use beamer only pass "xcolor=table" option, i.e. \documentclass[xcolor=table]{beamer}
\begin{table}[!htb]
    \centering
    \begin{tabular}{|
    >{\columncolor[HTML]{3166FF}}l |l|l|l|l|l|l|l|l|l|l|l|l|}
    \hline
    {\color[HTML]{FFFFFF} 12} &  &  &  &  &  &  &  &  &  &  &  &  \\ \hline
    {\color[HTML]{FFFFFF} 11} &  &  &  &  &  &  &  &  &  &  &  &  \\ \hline
    {\color[HTML]{FFFFFF} 10} &  &  &  & \cellcolor[HTML]{FE0000}B &  &  & \cellcolor[HTML]{FE0000}B &  &  &  &  &  \\ \hline
    {\color[HTML]{FFFFFF} 9} &  &  &  &  &  &  &  &  &  &  &  &  \\ \hline
    {\color[HTML]{FFFFFF} 8} &  &  &  &  &  &  &  &  & \cellcolor[HTML]{FE0000}B &  &  &  \\ \hline
    {\color[HTML]{FFFFFF} 7} &  &  &  &  &  &  &  &  &  &  &  &  \\ \hline
    {\color[HTML]{FFFFFF} 6} &  & \cellcolor[HTML]{34FF34}A &  &  &  &  &  &  &  &  &  &  \\ \hline
    {\color[HTML]{FFFFFF} 5} &  &  &  &  &  &  &  &  &  &  &  &  \\ \hline
    {\color[HTML]{FFFFFF} 4} &  &  &  & \cellcolor[HTML]{34FF34}A &  &  &  &  &  &  &  &  \\ \hline
    {\color[HTML]{FFFFFF} 3} &  &  &  &  &  & \cellcolor[HTML]{34FF34}A &  &  &  &  &  &  \\ \hline
    {\color[HTML]{FFFFFF} 2} &  &  &  &  &  &  &  &  &  &  &  &  \\ \hline
    {\color[HTML]{FFFFFF} 1} &  &  &  &  &  &  &  &  &  &  &  &  \\ \hline
    {\color[HTML]{FFFFFF} 0} & \cellcolor[HTML]{3166FF}{\color[HTML]{FFFFFF} 1} & \cellcolor[HTML]{3166FF}{\color[HTML]{FFFFFF} 2} & \cellcolor[HTML]{3166FF}{\color[HTML]{FFFFFF} 3} & \cellcolor[HTML]{3166FF}{\color[HTML]{FFFFFF} 4} & \cellcolor[HTML]{3166FF}{\color[HTML]{FFFFFF} 5} & \cellcolor[HTML]{3166FF}{\color[HTML]{FFFFFF} 6} & \cellcolor[HTML]{3166FF}{\color[HTML]{FFFFFF} 7} & \cellcolor[HTML]{3166FF}{\color[HTML]{FFFFFF} 8} & \cellcolor[HTML]{3166FF}{\color[HTML]{FFFFFF} 9} & \cellcolor[HTML]{3166FF}{\color[HTML]{FFFFFF} 10} & \cellcolor[HTML]{3166FF}{\color[HTML]{FFFFFF} 11} & \cellcolor[HTML]{3166FF}{\color[HTML]{FFFFFF} 12} \\ \hline
    \end{tabular}
\end{table}

\begin{itemize}
    \item Entrene una ADALINE con bias mediante la regla DELTA.  Proponga una vector de pesos iniciales $W_0=(w_1,w_2,w_3 )^T$ y un $\alpha$. Muestre la línea inicial, el vector de pesos final, así como la línea de separación final.
	\item Use la ADALINE  entrenado para determinar la clase A o B de los puntos con coordenadas (5,5) y (6,8).
\end{itemize}

En este ejercicio tuve varios problemas para resolverlo, pensé que era algunas cuestiones de la función de activación o del como se graficaba el resultado obtenido. Lo que realicé para resolver el problema fue cambiar los objetivos, en lugar de que fueran 0 y 1 como fue el caso del perceptrón pero con esto no era suficiente, entonces probé algo raro que sigo sin poder saber porque funcionó, lo que hice, fue cambiar el orden de los puntos a evaluar y la clasificación designada, dado que, yo los tenía en orden y no era posible resolver el ejercicio.


\newpage
\subsection{Desarrollo de Adaline}
\begin{tcolorbox}[breakable, size=fbox, boxrule=1pt, pad at break*=1mm,colback=cellbackground, colframe=cellborder]
\prompt{In}{incolor}{1}{\boxspacing}
\begin{Verbatim}[commandchars=\\\{\}]
\PY{c+c1}{\PYZsh{} Importando librerías}
\PY{k+kn}{import} \PY{n+nn}{numpy} \PY{k}{as} \PY{n+nn}{np}
\PY{k+kn}{import} \PY{n+nn}{matplotlib}\PY{n+nn}{.}\PY{n+nn}{pyplot} \PY{k}{as} \PY{n+nn}{plt}
\PY{k+kn}{import} \PY{n+nn}{time}
\PY{n}{err\PYZus{}vectorA} \PY{o}{=} \PY{p}{[}\PY{p}{]} \PY{c+c1}{\PYZsh{}vector de error}
\PY{n}{w\PYZus{}fA} \PY{o}{=} \PY{p}{[}\PY{p}{]} \PY{c+c1}{\PYZsh{}vector de pesos finales}
\end{Verbatim}
\end{tcolorbox}

    \begin{tcolorbox}[breakable, size=fbox, boxrule=1pt, pad at break*=1mm,colback=cellbackground, colframe=cellborder]
\prompt{In}{incolor}{2}{\boxspacing}
\begin{Verbatim}[commandchars=\\\{\}]
\PY{c+c1}{\PYZsh{}Entradas}
\PY{n}{x} \PY{o}{=} \PY{n}{np}\PY{o}{.}\PY{n}{array}\PY{p}{(}\PY{p}{[}\PY{p}{[}\PY{l+m+mi}{2}\PY{p}{,} \PY{l+m+mi}{6}\PY{p}{]}\PY{p}{,} \PY{p}{[}\PY{l+m+mi}{4}\PY{p}{,} \PY{l+m+mi}{4}\PY{p}{]}\PY{p}{,} \PY{p}{[}\PY{l+m+mi}{4}\PY{p}{,} \PY{l+m+mi}{10}\PY{p}{]}\PY{p}{,} \PY{p}{[}\PY{l+m+mi}{6}\PY{p}{,} \PY{l+m+mi}{3}\PY{p}{]}\PY{p}{,} \PY{p}{[}\PY{l+m+mi}{7}\PY{p}{,}\PY{l+m+mi}{10}\PY{p}{]}\PY{p}{,} \PY{p}{[}\PY{l+m+mi}{9}\PY{p}{,}\PY{l+m+mi}{8}\PY{p}{]}\PY{p}{]}\PY{p}{)}

\PY{n+nb}{print}\PY{p}{(}\PY{l+s+sa}{f}\PY{l+s+s2}{\PYZdq{}}\PY{l+s+se}{\PYZbs{}n}\PY{l+s+s2}{Entradas: }\PY{l+s+se}{\PYZbs{}n}\PY{l+s+si}{\PYZob{}}\PY{n}{x}\PY{l+s+si}{\PYZcb{}}\PY{l+s+s2}{\PYZdq{}}\PY{p}{)}
\PY{c+c1}{\PYZsh{}target}
\PY{n}{tA} \PY{o}{=} \PY{n}{np}\PY{o}{.}\PY{n}{array}\PY{p}{(}\PY{p}{[}\PY{p}{[}\PY{o}{\PYZhy{}}\PY{l+m+mi}{1}\PY{p}{]}\PY{p}{,} \PY{p}{[}\PY{o}{\PYZhy{}}\PY{l+m+mi}{1}\PY{p}{]}\PY{p}{,} \PY{p}{[}\PY{l+m+mi}{1}\PY{p}{]}\PY{p}{,} \PY{p}{[}\PY{o}{\PYZhy{}}\PY{l+m+mi}{1}\PY{p}{]}\PY{p}{,} \PY{p}{[}\PY{l+m+mi}{1}\PY{p}{]}\PY{p}{,} \PY{p}{[}\PY{l+m+mi}{1}\PY{p}{]}\PY{p}{]}\PY{p}{)}
\PY{n+nb}{print}\PY{p}{(}\PY{l+s+sa}{f}\PY{l+s+s2}{\PYZdq{}}\PY{l+s+se}{\PYZbs{}n}\PY{l+s+s2}{Objetivos: }\PY{l+s+se}{\PYZbs{}n}\PY{l+s+si}{\PYZob{}}\PY{n}{tA}\PY{l+s+si}{\PYZcb{}}\PY{l+s+s2}{\PYZdq{}}\PY{p}{)}
\PY{c+c1}{\PYZsh{}aprendizaje}
\PY{n}{alpha} \PY{o}{=} \PY{l+m+mf}{0.025}
\PY{n+nb}{print}\PY{p}{(}\PY{l+s+sa}{f}\PY{l+s+s2}{\PYZdq{}}\PY{l+s+s2}{Alpha:}\PY{l+s+se}{\PYZbs{}n}\PY{l+s+si}{\PYZob{}}\PY{n}{alpha}\PY{l+s+si}{\PYZcb{}}\PY{l+s+se}{\PYZbs{}n}\PY{l+s+s2}{\PYZdq{}}\PY{p}{)}

\PY{c+c1}{\PYZsh{}Se crea una columna de bias (\PYZhy{}1) del tamaño de columnas de la matriz de entrada }
\PY{c+c1}{\PYZsh{}Se establece el tamaño del bias que debe ser del numero de columas que tiene x}
\PY{n}{biasA} \PY{o}{=} \PY{n}{np}\PY{o}{.}\PY{n}{shape}\PY{p}{(}\PY{n}{x}\PY{p}{)}\PY{p}{[}\PY{l+m+mi}{0}\PY{p}{]} \PY{c+c1}{\PYZsh{}[0] \PYZhy{}\PYZgt{} columnas      [1] \PYZhy{}\PYZgt{} filas}
\PY{n}{biasA} \PY{o}{=} \PY{o}{\PYZhy{}}\PY{l+m+mi}{1}\PY{o}{*}\PY{n}{np}\PY{o}{.}\PY{n}{ones}\PY{p}{(}\PY{p}{(}\PY{n}{biasA}\PY{p}{,} \PY{l+m+mi}{1}\PY{p}{)}\PY{p}{)} \PY{c+c1}{\PYZsh{} np.ones((filas,columnas))}
\PY{n+nb}{print}\PY{p}{(}\PY{l+s+sa}{f}\PY{l+s+s2}{\PYZdq{}}\PY{l+s+s2}{Bias:}\PY{l+s+se}{\PYZbs{}n}\PY{l+s+si}{\PYZob{}}\PY{n}{biasA}\PY{l+s+si}{\PYZcb{}}\PY{l+s+se}{\PYZbs{}n}\PY{l+s+s2}{\PYZdq{}}\PY{p}{)}

\PY{c+c1}{\PYZsh{}concatenar x con bias \PYZdq{}Vector Aumentado}
\PY{n}{x} \PY{o}{=} \PY{n}{np}\PY{o}{.}\PY{n}{concatenate}\PY{p}{(}\PY{p}{[}\PY{n}{x}\PY{p}{,} \PY{n}{biasA}\PY{p}{]}\PY{p}{,} \PY{n}{axis} \PY{o}{=} \PY{l+m+mi}{1}\PY{p}{)} \PY{c+c1}{\PYZsh{}1 \PYZhy{}\PYZgt{} añade a la derecha   0 \PYZhy{}\PYZgt{} añade abajo}
\PY{n+nb}{print}\PY{p}{(}\PY{l+s+sa}{f}\PY{l+s+s2}{\PYZdq{}}\PY{l+s+s2}{Matriz Aumentada:}\PY{l+s+se}{\PYZbs{}n}\PY{l+s+si}{\PYZob{}}\PY{n}{x}\PY{l+s+si}{\PYZcb{}}\PY{l+s+se}{\PYZbs{}n}\PY{l+s+s2}{\PYZdq{}}\PY{p}{)}

\PY{c+c1}{\PYZsh{}matriz de pesos}
\PY{n}{w\PYZus{}iA} \PY{o}{=} \PY{n}{np}\PY{o}{.}\PY{n}{array}\PY{p}{(}\PY{p}{[}\PY{p}{[}\PY{o}{\PYZhy{}}\PY{l+m+mf}{0.25}\PY{p}{]}\PY{p}{,}\PY{p}{[}\PY{l+m+mf}{0.25}\PY{p}{]}\PY{p}{,} \PY{p}{[}\PY{o}{\PYZhy{}}\PY{l+m+mf}{0.5}\PY{p}{]}\PY{p}{]}\PY{p}{)}

\PY{n+nb}{print}\PY{p}{(}\PY{l+s+sa}{f}\PY{l+s+s2}{\PYZdq{}}\PY{l+s+s2}{Pesos:}\PY{l+s+se}{\PYZbs{}n}\PY{l+s+si}{\PYZob{}}\PY{n}{w\PYZus{}iA}\PY{l+s+si}{\PYZcb{}}\PY{l+s+se}{\PYZbs{}n}\PY{l+s+s2}{\PYZdq{}}\PY{p}{)}
\end{Verbatim}
\end{tcolorbox}

    \begin{Verbatim}[commandchars=\\\{\}]

Entradas:
[[ 2  6]
 [ 4  4]
 [ 4 10]
 [ 6  3]
 [ 7 10]
 [ 9  8]]

Objetivos:
[[-1]
 [-1]
 [ 1]
 [-1]
 [ 1]
 [ 1]]
Alpha:
0.025

Bias:
[[-1.]
 [-1.]
 [-1.]
 [-1.]
 [-1.]
 [-1.]]

Matriz Aumentada:
[[ 2.  6. -1.]
 [ 4.  4. -1.]
 [ 4. 10. -1.]
 [ 6.  3. -1.]
 [ 7. 10. -1.]
 [ 9.  8. -1.]]

Pesos:
[[-0.25]
 [ 0.25]
 [-0.5 ]]

    \end{Verbatim}

    \begin{tcolorbox}[breakable, size=fbox, boxrule=1pt, pad at break*=1mm,colback=cellbackground, colframe=cellborder]
\prompt{In}{incolor}{3}{\boxspacing}
\begin{Verbatim}[commandchars=\\\{\}]
\PY{c+c1}{\PYZsh{}\PYZpc{}\PYZpc{} Regla Delta}
\PY{n}{ticA} \PY{o}{=} \PY{n}{time}\PY{o}{.}\PY{n}{time}\PY{p}{(}\PY{p}{)} \PY{c+c1}{\PYZsh{}Iniciamos cronometro}

\PY{c+c1}{\PYZsh{} Producto vectorial punto}
\PY{n}{a} \PY{o}{=} \PY{n}{np}\PY{o}{.}\PY{n}{dot}\PY{p}{(}\PY{n}{x}\PY{p}{,} \PY{n}{w\PYZus{}iA}\PY{p}{)}
\PY{n+nb}{print}\PY{p}{(}\PY{l+s+sa}{f}\PY{l+s+s2}{\PYZdq{}}\PY{l+s+s2}{a:}\PY{l+s+se}{\PYZbs{}n}\PY{l+s+si}{\PYZob{}}\PY{n}{a}\PY{l+s+si}{\PYZcb{}}\PY{l+s+se}{\PYZbs{}n}\PY{l+s+s2}{\PYZdq{}}\PY{p}{)}

\PY{c+c1}{\PYZsh{}Funcion Lineal}
\PY{n}{yA} \PY{o}{=} \PY{n}{a}
\PY{n+nb}{print}\PY{p}{(}\PY{l+s+sa}{f}\PY{l+s+s2}{\PYZdq{}}\PY{l+s+s2}{y:}\PY{l+s+se}{\PYZbs{}n}\PY{l+s+si}{\PYZob{}}\PY{n}{yA}\PY{l+s+si}{\PYZcb{}}\PY{l+s+se}{\PYZbs{}n}\PY{l+s+s2}{\PYZdq{}}\PY{p}{)}

\PY{c+c1}{\PYZsh{} Función de costo \PYZhy{}\PYZhy{} MSE Error cuadrático medio}
\PY{n}{errA} \PY{o}{=} \PY{n}{np}\PY{o}{.}\PY{n}{sum}\PY{p}{(}\PY{l+m+mf}{0.5} \PY{o}{*} \PY{p}{(}\PY{n}{tA} \PY{o}{\PYZhy{}} \PY{n}{yA}\PY{p}{)} \PY{o}{*}\PY{o}{*} \PY{l+m+mi}{2}\PY{p}{)}
\PY{n}{err\PYZus{}vectorA}\PY{o}{.}\PY{n}{append}\PY{p}{(}\PY{n}{errA}\PY{p}{)}
\PY{n+nb}{print}\PY{p}{(}\PY{l+s+sa}{f}\PY{l+s+s2}{\PYZdq{}}\PY{l+s+s2}{Error inicial:}\PY{l+s+se}{\PYZbs{}n}\PY{l+s+si}{\PYZob{}}\PY{n}{errA}\PY{l+s+si}{\PYZcb{}}\PY{l+s+se}{\PYZbs{}n}\PY{l+s+s2}{\PYZdq{}}\PY{p}{)}

\PY{n}{epochA} \PY{o}{=} \PY{l+m+mi}{0}
\PY{n}{epocas} \PY{o}{=} \PY{l+m+mi}{500}
\end{Verbatim}
\end{tcolorbox}

    \begin{Verbatim}[commandchars=\\\{\}]
a:
[[ 1.5 ]
 [ 0.5 ]
 [ 2.  ]
 [-0.25]
 [ 1.25]
 [ 0.25]]

y:
[[ 1.5 ]
 [ 0.5 ]
 [ 2.  ]
 [-0.25]
 [ 1.25]
 [ 0.25]]

Error inicial:
5.34375

    \end{Verbatim}

    \begin{tcolorbox}[breakable, size=fbox, boxrule=1pt, pad at break*=1mm,colback=cellbackground, colframe=cellborder]
\prompt{In}{incolor}{4}{\boxspacing}
\begin{Verbatim}[commandchars=\\\{\}]
\PY{k}{for} \PY{n}{i} \PY{o+ow}{in} \PY{n+nb}{range}\PY{p}{(}\PY{n}{epocas}\PY{p}{)}\PY{p}{:}
  \PY{n}{epochA} \PY{o}{+}\PY{o}{=} \PY{l+m+mi}{1}
  \PY{c+c1}{\PYZsh{}print(f\PYZdq{}Epoca: \PYZob{}epochA\PYZcb{}\PYZbs{}n\PYZdq{})}

  \PY{k}{for} \PY{n}{i} \PY{o+ow}{in} \PY{n+nb}{range}\PY{p}{(}\PY{n}{np}\PY{o}{.}\PY{n}{shape}\PY{p}{(}\PY{n}{x}\PY{p}{)}\PY{p}{[}\PY{l+m+mi}{0}\PY{p}{]}\PY{p}{)}\PY{p}{:} \PY{c+c1}{\PYZsh{}Repetir segun la cantidad de filas de x}
    \PY{c+c1}{\PYZsh{} Producto punto}
    \PY{n}{a} \PY{o}{=} \PY{n}{np}\PY{o}{.}\PY{n}{dot}\PY{p}{(}\PY{n}{x}\PY{p}{[}\PY{n}{i}\PY{p}{]}\PY{p}{,} \PY{n}{w\PYZus{}iA}\PY{p}{)}

    \PY{c+c1}{\PYZsh{}Funcion lineal}
    \PY{n}{yA}\PY{p}{[}\PY{n}{i}\PY{p}{]} \PY{o}{=} \PY{n}{a}

    \PY{c+c1}{\PYZsh{} Actualización de pesos}
    \PY{n}{x\PYZus{}T} \PY{o}{=} \PY{n}{np}\PY{o}{.}\PY{n}{reshape}\PY{p}{(}\PY{n}{x}\PY{p}{[}\PY{n}{i}\PY{p}{]}\PY{p}{,} \PY{p}{(}\PY{n+nb}{len}\PY{p}{(}\PY{n}{w\PYZus{}iA}\PY{p}{)}\PY{p}{,} \PY{l+m+mi}{1}\PY{p}{)}\PY{p}{)} \PY{c+c1}{\PYZsh{}REACOMODA np.reshape(matriz,(filas,columnas))}
    \PY{n}{w\PYZus{}nA} \PY{o}{=} \PY{n}{w\PYZus{}iA} \PY{o}{+} \PY{n}{alpha} \PY{o}{*} \PY{p}{(}\PY{n}{tA}\PY{p}{[}\PY{n}{i}\PY{p}{]} \PY{o}{\PYZhy{}} \PY{n}{yA}\PY{p}{[}\PY{n}{i}\PY{p}{]}\PY{p}{)} \PY{o}{*} \PY{n}{x\PYZus{}T}
    \PY{n}{w\PYZus{}iA} \PY{o}{=} \PY{n}{w\PYZus{}nA}
    \PY{c+c1}{\PYZsh{}print(f\PYZdq{}Peso nuevo:\PYZbs{}n\PYZob{}w\PYZus{}i\PYZcb{}\PYZbs{}n\PYZdq{})}
 
  \PY{c+c1}{\PYZsh{} Función de costo }

  \PY{n}{errA} \PY{o}{=} \PY{p}{(}\PY{n}{np}\PY{o}{.}\PY{n}{sum}\PY{p}{(}\PY{l+m+mf}{0.5} \PY{o}{*} \PY{p}{(}\PY{n}{tA} \PY{o}{\PYZhy{}} \PY{n}{yA}\PY{p}{)}\PY{o}{*}\PY{o}{*}\PY{l+m+mi}{2}\PY{p}{)}\PY{p}{)} \PY{o}{/} \PY{n+nb}{len}\PY{p}{(}\PY{n}{tA}\PY{p}{)}
  \PY{n}{err\PYZus{}vectorA}\PY{o}{.}\PY{n}{append}\PY{p}{(}\PY{n}{errA}\PY{p}{)}
  \PY{c+c1}{\PYZsh{}print(f\PYZdq{}Error:\PYZbs{}n\PYZob{}errA\PYZcb{}\PYZbs{}n\PYZdq{})}
  \PY{c+c1}{\PYZsh{}Se añade al vector de pesos}
  \PY{n}{w\PYZus{}fA}\PY{o}{.}\PY{n}{append}\PY{p}{(}\PY{n}{w\PYZus{}iA}\PY{p}{)}
  \PY{c+c1}{\PYZsh{}print(f\PYZdq{}Vector de Pesos Finales:\PYZbs{}n\PYZob{}w\PYZus{}fA\PYZcb{}\PYZbs{}n\PYZdq{})}

\PY{n+nb}{print}\PY{p}{(}\PY{l+s+s2}{\PYZdq{}}\PY{l+s+se}{\PYZbs{}t}\PY{l+s+s2}{Entrenamiento Adaline}\PY{l+s+se}{\PYZbs{}n}\PY{l+s+s2}{\PYZdq{}}\PY{p}{)}
\PY{n+nb}{print}\PY{p}{(}\PY{l+s+sa}{f}\PY{l+s+s2}{\PYZdq{}}\PY{l+s+s2}{Error final:}\PY{l+s+se}{\PYZbs{}n}\PY{l+s+si}{\PYZob{}}\PY{n}{errA}\PY{l+s+si}{\PYZcb{}}\PY{l+s+se}{\PYZbs{}n}\PY{l+s+s2}{\PYZdq{}}\PY{p}{)}
\PY{n+nb}{print}\PY{p}{(}\PY{l+s+sa}{f}\PY{l+s+s2}{\PYZdq{}}\PY{l+s+s2}{Vector de Pesos Finales:}\PY{l+s+se}{\PYZbs{}n}\PY{l+s+si}{\PYZob{}}\PY{n}{w\PYZus{}iA}\PY{l+s+si}{\PYZcb{}}\PY{l+s+se}{\PYZbs{}n}\PY{l+s+s2}{\PYZdq{}}\PY{p}{)}

\PY{n}{tocA} \PY{o}{=} \PY{n}{time}\PY{o}{.}\PY{n}{time}\PY{p}{(}\PY{p}{)} \PY{c+c1}{\PYZsh{}Paro cronometro}
\end{Verbatim}
\end{tcolorbox}

    \begin{Verbatim}[commandchars=\\\{\}]
        Entrenamiento Adaline

Error final:
0.09294818214166724

Vector de Pesos Finales:
[[0.50784435]
 [0.30789657]
 [4.50158229]]

    \end{Verbatim}

    \begin{tcolorbox}[breakable, size=fbox, boxrule=1pt, pad at break*=1mm,colback=cellbackground, colframe=cellborder]
\prompt{In}{incolor}{5}{\boxspacing}
\begin{Verbatim}[commandchars=\\\{\}]
\PY{c+c1}{\PYZsh{}\PYZpc{}\PYZpc{} Plotting Error \PYZhy{}\PYZhy{} Graph}
\PY{n}{plt}\PY{o}{.}\PY{n}{figure}\PY{p}{(}\PY{l+m+mi}{0}\PY{p}{)}
\PY{n}{plt}\PY{o}{.}\PY{n}{plot}\PY{p}{(}\PY{n}{err\PYZus{}vectorA}\PY{p}{,} \PY{n}{linewidth} \PY{o}{=} \PY{l+m+mi}{2}\PY{p}{)}
\PY{n}{plt}\PY{o}{.}\PY{n}{title}\PY{p}{(}\PY{l+s+s1}{\PYZsq{}}\PY{l+s+s1}{Gráfica de error: REGLA DELTA}\PY{l+s+s1}{\PYZsq{}}\PY{p}{)}
\PY{n}{plt}\PY{o}{.}\PY{n}{ylabel}\PY{p}{(}\PY{l+s+s1}{\PYZsq{}}\PY{l+s+s1}{Magnitud de Error}\PY{l+s+s1}{\PYZsq{}}\PY{p}{)}
\PY{n}{plt}\PY{o}{.}\PY{n}{xlabel}\PY{p}{(}\PY{l+s+s1}{\PYZsq{}}\PY{l+s+s1}{Épocas}\PY{l+s+s1}{\PYZsq{}}\PY{p}{)}
\PY{n}{plt}\PY{o}{.}\PY{n}{scatter}\PY{p}{(}\PY{n+nb}{len}\PY{p}{(}\PY{n}{err\PYZus{}vectorA}\PY{p}{)} \PY{o}{\PYZhy{}} \PY{l+m+mi}{1}\PY{p}{,} \PY{l+m+mi}{0}\PY{p}{,} \PY{n}{color} \PY{o}{=} \PY{l+s+s1}{\PYZsq{}}\PY{l+s+s1}{r}\PY{l+s+s1}{\PYZsq{}}\PY{p}{,} \PY{n}{s} \PY{o}{=} \PY{l+m+mi}{200}\PY{p}{,} \PY{n}{marker} \PY{o}{=} \PY{l+s+s1}{\PYZsq{}}\PY{l+s+s1}{o}\PY{l+s+s1}{\PYZsq{}}\PY{p}{,} \PY{n}{alpha} \PY{o}{=} \PY{l+m+mf}{0.4}\PY{p}{)}
\PY{n}{plt}\PY{o}{.}\PY{n}{show}\PY{p}{(}\PY{p}{)}
\end{Verbatim}
\end{tcolorbox}

    \begin{center}
    \adjustimage{max size={0.9\linewidth}{0.9\paperheight}}{sections/adaline/output_4_0.png}
    \end{center}
    { \hspace*{\fill} \\}
    
    \begin{tcolorbox}[breakable, size=fbox, boxrule=1pt, pad at break*=1mm,colback=cellbackground, colframe=cellborder]
\prompt{In}{incolor}{6}{\boxspacing}
\begin{Verbatim}[commandchars=\\\{\}]
\PY{c+c1}{\PYZsh{}Plotting Decision Boundaries}
\PY{n}{plt}\PY{o}{.}\PY{n}{xlim}\PY{p}{(}\PY{p}{[}\PY{o}{\PYZhy{}}\PY{l+m+mf}{1.0}\PY{p}{,} \PY{l+m+mf}{15.0}\PY{p}{]}\PY{p}{)}
\PY{n}{plt}\PY{o}{.}\PY{n}{ylim}\PY{p}{(}\PY{p}{[}\PY{o}{\PYZhy{}}\PY{l+m+mf}{1.0}\PY{p}{,} \PY{l+m+mf}{15.0}\PY{p}{]}\PY{p}{)}

\PY{n}{patterns} \PY{o}{=} \PY{n}{np}\PY{o}{.}\PY{n}{unique}\PY{p}{(}\PY{n}{tA}\PY{p}{)} \PY{c+c1}{\PYZsh{}Encuentra los elementos únicos de la matriz t}

\PY{k}{for} \PY{n}{patt} \PY{o+ow}{in} \PY{n}{patterns}\PY{p}{:}
  \PY{n}{pos} \PY{o}{=} \PY{n}{np}\PY{o}{.}\PY{n}{where}\PY{p}{(}\PY{n}{patt} \PY{o}{==} \PY{n}{tA}\PY{p}{)}\PY{p}{[}\PY{l+m+mi}{0}\PY{p}{]} \PY{c+c1}{\PYZsh{} np.where(TRUE)[0]}
  \PY{k}{if} \PY{n}{patt} \PY{o}{==} \PY{o}{\PYZhy{}}\PY{l+m+mi}{1}\PY{p}{:}
    \PY{n}{plt}\PY{o}{.}\PY{n}{scatter}\PY{p}{(}\PY{n}{x}\PY{p}{[}\PY{n}{pos}\PY{p}{,} \PY{l+m+mi}{0}\PY{p}{]}\PY{p}{,} \PY{n}{x}\PY{p}{[}\PY{n}{pos}\PY{p}{,} \PY{l+m+mi}{1}\PY{p}{]}\PY{p}{,} \PY{n}{color} \PY{o}{=} \PY{l+s+s1}{\PYZsq{}}\PY{l+s+s1}{g}\PY{l+s+s1}{\PYZsq{}}\PY{p}{,} \PY{n}{s} \PY{o}{=} \PY{l+m+mi}{200}\PY{p}{,} \PY{n}{marker} \PY{o}{=} \PY{l+s+s1}{\PYZsq{}}\PY{l+s+s1}{o}\PY{l+s+s1}{\PYZsq{}}\PY{p}{,} \PY{n}{alpha} \PY{o}{=} \PY{l+m+mf}{0.8}\PY{p}{)}
  \PY{k}{else}\PY{p}{:}
    \PY{n}{plt}\PY{o}{.}\PY{n}{scatter}\PY{p}{(}\PY{n}{x}\PY{p}{[}\PY{n}{pos}\PY{p}{,} \PY{l+m+mi}{0}\PY{p}{]}\PY{p}{,} \PY{n}{x}\PY{p}{[}\PY{n}{pos}\PY{p}{,} \PY{l+m+mi}{1}\PY{p}{]}\PY{p}{,} \PY{n}{color} \PY{o}{=} \PY{l+s+s1}{\PYZsq{}}\PY{l+s+s1}{b}\PY{l+s+s1}{\PYZsq{}}\PY{p}{,} \PY{n}{s} \PY{o}{=} \PY{l+m+mi}{200}\PY{p}{,} \PY{n}{marker} \PY{o}{=} \PY{l+s+s1}{\PYZsq{}}\PY{l+s+s1}{x}\PY{l+s+s1}{\PYZsq{}}\PY{p}{,} \PY{n}{alpha} \PY{o}{=} \PY{l+m+mf}{0.8}\PY{p}{)}

\PY{n}{x1A} \PY{o}{=} \PY{n}{np}\PY{o}{.}\PY{n}{linspace}\PY{p}{(}\PY{o}{\PYZhy{}}\PY{l+m+mi}{1}\PY{p}{,} \PY{l+m+mi}{15}\PY{p}{)}
\PY{n}{x2A} \PY{o}{=} \PY{n}{w\PYZus{}iA}\PY{p}{[}\PY{l+m+mi}{2}\PY{p}{]} \PY{o}{/} \PY{n}{w\PYZus{}iA}\PY{p}{[}\PY{l+m+mi}{1}\PY{p}{]} \PY{o}{\PYZhy{}} \PY{p}{(}\PY{n}{x1A} \PY{o}{*} \PY{n}{w\PYZus{}iA}\PY{p}{[}\PY{l+m+mi}{0}\PY{p}{]}\PY{p}{)} \PY{o}{/} \PY{n}{w\PYZus{}iA}\PY{p}{[}\PY{l+m+mi}{1}\PY{p}{]}


\PY{n}{plt}\PY{o}{.}\PY{n}{figure}\PY{p}{(}\PY{l+m+mi}{1}\PY{p}{)}
\PY{n}{plt}\PY{o}{.}\PY{n}{plot}\PY{p}{(}\PY{n}{x1A}\PY{p}{,} \PY{n}{x2A}\PY{p}{,} \PY{l+s+s1}{\PYZsq{}}\PY{l+s+s1}{red}\PY{l+s+s1}{\PYZsq{}}\PY{p}{,} \PY{n}{linewidth} \PY{o}{=} \PY{l+m+mi}{2}\PY{p}{)}
\PY{n}{plt}\PY{o}{.}\PY{n}{title}\PY{p}{(}\PY{l+s+s1}{\PYZsq{}}\PY{l+s+s1}{Fronteras de decisión: REGLA DE DELTA}\PY{l+s+s1}{\PYZsq{}}\PY{p}{)}
\PY{n}{plt}\PY{o}{.}\PY{n}{show}\PY{p}{(}\PY{p}{)}
\end{Verbatim}
\end{tcolorbox}

    \begin{center}
    \adjustimage{max size={0.9\linewidth}{0.9\paperheight}}{sections/adaline/output_5_0.png}
    \end{center}
    { \hspace*{\fill} \\}
    
    \begin{tcolorbox}[breakable, size=fbox, boxrule=1pt, pad at break*=1mm,colback=cellbackground, colframe=cellborder]
\prompt{In}{incolor}{7}{\boxspacing}
\begin{Verbatim}[commandchars=\\\{\}]
\PY{c+c1}{\PYZsh{}Pesos Finales}
\PY{n+nb}{print}\PY{p}{(}\PY{l+s+s1}{\PYZsq{}}\PY{l+s+se}{\PYZbs{}n}\PY{l+s+s1}{Pesos finales: }\PY{l+s+s1}{\PYZsq{}}\PY{p}{)}
\PY{k}{for} \PY{n}{i} \PY{o+ow}{in} \PY{n+nb}{range}\PY{p}{(}\PY{l+m+mi}{1}\PY{p}{)}\PY{p}{:}
    \PY{n}{resA} \PY{o}{=} \PY{n+nb}{str}\PY{p}{(}\PY{n}{w\PYZus{}iA}\PY{p}{)}
    \PY{n+nb}{print}\PY{p}{(}\PY{n}{resA}\PY{p}{)}
    \PY{n+nb}{print}\PY{p}{(}\PY{p}{)}

\PY{c+c1}{\PYZsh{}\PYZpc{}\PYZpc{} Displaying Results}
\PY{n}{aA} \PY{o}{=} \PY{n}{np}\PY{o}{.}\PY{n}{dot}\PY{p}{(}\PY{n}{x}\PY{p}{,} \PY{n}{w\PYZus{}iA}\PY{p}{)}
\PY{n}{yA}\PY{o}{=} \PY{n}{aA}


\PY{n+nb}{print}\PY{p}{(}\PY{l+s+s1}{\PYZsq{}}\PY{l+s+s1}{REGLA DELTA}\PY{l+s+s1}{\PYZsq{}}\PY{p}{)}
\PY{n+nb}{print}\PY{p}{(}\PY{l+s+s1}{\PYZsq{}}\PY{l+s+s1}{Meta:    Predicción:}\PY{l+s+s1}{\PYZsq{}}\PY{p}{)}
\PY{k}{for} \PY{n}{i} \PY{o+ow}{in} \PY{n+nb}{range}\PY{p}{(}\PY{n+nb}{len}\PY{p}{(}\PY{n}{yA}\PY{p}{)}\PY{p}{)}\PY{p}{:}
    \PY{n}{resA} \PY{o}{=} \PY{n+nb}{str}\PY{p}{(}\PY{n}{tA}\PY{p}{[}\PY{n}{i}\PY{p}{]}\PY{p}{)} \PY{o}{+} \PY{l+s+s1}{\PYZsq{}}\PY{l+s+s1}{\PYZhy{}\PYZhy{}\PYZhy{}\PYZhy{}\PYZhy{}\PYZhy{}\PYZhy{}\PYZhy{}}\PY{l+s+s1}{\PYZsq{}} \PY{o}{+} \PY{n+nb}{str}\PY{p}{(}\PY{n}{yA}\PY{p}{[}\PY{n}{i}\PY{p}{]}\PY{p}{)}
    \PY{n+nb}{print}\PY{p}{(}\PY{n}{resA}\PY{p}{)}

\PY{n+nb}{print}\PY{p}{(}\PY{l+s+sa}{f}\PY{l+s+s1}{\PYZsq{}}\PY{l+s+se}{\PYZbs{}n}\PY{l+s+s1}{Tiempo requerido ADALINE: }\PY{l+s+si}{\PYZob{}}\PY{n}{tocA} \PY{o}{\PYZhy{}} \PY{n}{ticA}\PY{l+s+si}{:}\PY{l+s+s1}{.5f}\PY{l+s+si}{\PYZcb{}}\PY{l+s+s1}{ ms.}\PY{l+s+s1}{\PYZsq{}}\PY{p}{)}
\PY{n+nb}{print}\PY{p}{(}\PY{l+s+sa}{f}\PY{l+s+s1}{\PYZsq{}}\PY{l+s+se}{\PYZbs{}n}\PY{l+s+s1}{Épocas requeridas: }\PY{l+s+si}{\PYZob{}}\PY{n}{epochA}\PY{l+s+si}{\PYZcb{}}\PY{l+s+s1}{.}\PY{l+s+se}{\PYZbs{}n}\PY{l+s+s1}{\PYZsq{}}\PY{p}{)}
\end{Verbatim}
\end{tcolorbox}

    \begin{Verbatim}[commandchars=\\\{\}]

Pesos finales:
[[0.50784435]
 [0.30789657]
 [4.50158229]]

REGLA DELTA
Meta:    Predicción:
[-1]--------[-1.63851418]
[-1]--------[-1.23861862]
[1]--------[0.60876079]
[-1]--------[-0.53082649]
[1]--------[2.13229384]
[1]--------[2.5321894]

Tiempo requerido ADALINE: 0.07000 ms.

Épocas requeridas: 500.

    \end{Verbatim}

    \begin{tcolorbox}[breakable, size=fbox, boxrule=1pt, pad at break*=1mm,colback=cellbackground, colframe=cellborder]
\prompt{In}{incolor}{8}{\boxspacing}
\begin{Verbatim}[commandchars=\\\{\}]
\PY{c+c1}{\PYZsh{}Prueba}
\PY{c+c1}{\PYZsh{} Plotting Decision Boundaries}
\PY{n}{plt}\PY{o}{.}\PY{n}{xlim}\PY{p}{(}\PY{p}{[}\PY{o}{\PYZhy{}}\PY{l+m+mf}{1.0}\PY{p}{,} \PY{l+m+mf}{15.0}\PY{p}{]}\PY{p}{)}
\PY{n}{plt}\PY{o}{.}\PY{n}{ylim}\PY{p}{(}\PY{p}{[}\PY{o}{\PYZhy{}}\PY{l+m+mf}{1.0}\PY{p}{,} \PY{l+m+mf}{15.0}\PY{p}{]}\PY{p}{)}


\PY{n}{plt}\PY{o}{.}\PY{n}{scatter}\PY{p}{(}\PY{l+m+mi}{5}\PY{p}{,} \PY{l+m+mi}{5}\PY{p}{,} \PY{n}{color} \PY{o}{=} \PY{l+s+s1}{\PYZsq{}}\PY{l+s+s1}{g}\PY{l+s+s1}{\PYZsq{}}\PY{p}{,} \PY{n}{s} \PY{o}{=} \PY{l+m+mi}{200}\PY{p}{,} \PY{n}{marker} \PY{o}{=} \PY{l+s+s1}{\PYZsq{}}\PY{l+s+s1}{o}\PY{l+s+s1}{\PYZsq{}}\PY{p}{,} \PY{n}{alpha} \PY{o}{=} \PY{l+m+mf}{0.8}\PY{p}{)}
\PY{n}{plt}\PY{o}{.}\PY{n}{scatter}\PY{p}{(}\PY{l+m+mi}{6}\PY{p}{,} \PY{l+m+mi}{8}\PY{p}{,} \PY{n}{color} \PY{o}{=} \PY{l+s+s1}{\PYZsq{}}\PY{l+s+s1}{b}\PY{l+s+s1}{\PYZsq{}}\PY{p}{,} \PY{n}{s} \PY{o}{=} \PY{l+m+mi}{200}\PY{p}{,} \PY{n}{marker} \PY{o}{=} \PY{l+s+s1}{\PYZsq{}}\PY{l+s+s1}{x}\PY{l+s+s1}{\PYZsq{}}\PY{p}{,} \PY{n}{alpha} \PY{o}{=} \PY{l+m+mf}{0.8}\PY{p}{)}

\PY{n}{x1} \PY{o}{=} \PY{n}{np}\PY{o}{.}\PY{n}{linspace}\PY{p}{(}\PY{o}{\PYZhy{}}\PY{l+m+mi}{1}\PY{p}{,} \PY{l+m+mi}{15}\PY{p}{)}
\PY{n}{x2} \PY{o}{=} \PY{n}{w\PYZus{}iA}\PY{p}{[}\PY{l+m+mi}{2}\PY{p}{]} \PY{o}{/} \PY{n}{w\PYZus{}iA}\PY{p}{[}\PY{l+m+mi}{1}\PY{p}{]} \PY{o}{\PYZhy{}} \PY{p}{(}\PY{n}{x1} \PY{o}{*} \PY{n}{w\PYZus{}iA}\PY{p}{[}\PY{l+m+mi}{0}\PY{p}{]}\PY{p}{)} \PY{o}{/} \PY{n}{w\PYZus{}iA}\PY{p}{[}\PY{l+m+mi}{1}\PY{p}{]}

\PY{n}{plt}\PY{o}{.}\PY{n}{plot}\PY{p}{(}\PY{n}{x1}\PY{p}{,} \PY{n}{x2}\PY{p}{,} \PY{l+s+s1}{\PYZsq{}}\PY{l+s+s1}{red}\PY{l+s+s1}{\PYZsq{}}\PY{p}{,} \PY{n}{linewidth} \PY{o}{=} \PY{l+m+mi}{2}\PY{p}{)}
\PY{n}{plt}\PY{o}{.}\PY{n}{title}\PY{p}{(}\PY{l+s+s1}{\PYZsq{}}\PY{l+s+s1}{Prueba de Adaline}\PY{l+s+s1}{\PYZsq{}}\PY{p}{)}
\PY{n}{plt}\PY{o}{.}\PY{n}{show}\PY{p}{(}\PY{p}{)}
\end{Verbatim}
\end{tcolorbox}

    \begin{center}
    \adjustimage{max size={0.9\linewidth}{0.9\paperheight}}{sections/adaline/output_7_0.png}
    \end{center}
    { \hspace*{\fill} \\}
    

    % Add a bibliography block to the postdoc

\clearpage