\section{RNA Compuesta}
Dada la siguiente tabla con dos clases A y B:

% Please add the following required packages to your document preamble:
% \usepackage[table,xcdraw]{xcolor}
% If you use beamer only pass "xcolor=table" option, i.e. \documentclass[xcolor=table]{beamer}
\begin{table}[!htb]
    \centering
    \begin{tabular}{|
    >{\columncolor[HTML]{3166FF}}l |l|l|l|l|l|l|l|l|l|l|l|l|}
    \hline
    {\color[HTML]{FFFFFF} 12} &  &  &  &  &  &  &  &  &  &  &  &  \\ \hline
    {\color[HTML]{FFFFFF} 11} &  &  &  &  &  & \cellcolor[HTML]{32CB00}{\color[HTML]{FFFFFF} A} &  &  &  &  &  &  \\ \hline
    {\color[HTML]{FFFFFF} 10} &  &  &  & \cellcolor[HTML]{FFFFFF} &  &  & \cellcolor[HTML]{FFFFFF} &  &  &  &  &  \\ \hline
    {\color[HTML]{FFFFFF} 9} &  &  & \cellcolor[HTML]{32CB00}{\color[HTML]{FFFFFF} A} &  &  &  &  & \cellcolor[HTML]{FE0000}{\color[HTML]{FFFFFF} B} &  & \cellcolor[HTML]{FE0000}{\color[HTML]{FFFFFF} B} &  &  \\ \hline
    {\color[HTML]{FFFFFF} 8} &  &  &  &  &  &  &  &  & \cellcolor[HTML]{FFFFFF} &  &  &  \\ \hline
    {\color[HTML]{FFFFFF} 7} &  & \cellcolor[HTML]{32CB00}{\color[HTML]{FFFFFF} A} &  &  & \cellcolor[HTML]{FE0000}{\color[HTML]{FFFFFF} B} &  &  &  &  &  &  &  \\ \hline
    {\color[HTML]{FFFFFF} 6} &  & \cellcolor[HTML]{FFFFFF} &  &  &  &  &  & \cellcolor[HTML]{FE0000}{\color[HTML]{FFFFFF} B} &  & \cellcolor[HTML]{FE0000}{\color[HTML]{FFFFFF} B} &  &  \\ \hline
    {\color[HTML]{FFFFFF} 5} &  &  &  &  &  &  &  &  &  &  &  &  \\ \hline
    {\color[HTML]{FFFFFF} 4} &  &  &  & \cellcolor[HTML]{32CB00}{\color[HTML]{FFFFFF} A} &  &  & \cellcolor[HTML]{32CB00}{\color[HTML]{FFFFFF} A} &  &  &  &  &  \\ \hline
    {\color[HTML]{FFFFFF} 3} &  &  &  &  &  & \cellcolor[HTML]{FFFFFF} &  &  &  &  &  &  \\ \hline
    {\color[HTML]{FFFFFF} 2} &  &  &  &  &  &  &  &  &  &  &  &  \\ \hline
    {\color[HTML]{FFFFFF} 1} &  &  &  &  &  &  &  &  &  &  &  &  \\ \hline
    {\color[HTML]{FFFFFF} 0} & \cellcolor[HTML]{3166FF}{\color[HTML]{FFFFFF} 1} & \cellcolor[HTML]{3166FF}{\color[HTML]{FFFFFF} 2} & \cellcolor[HTML]{3166FF}{\color[HTML]{FFFFFF} 3} & \cellcolor[HTML]{3166FF}{\color[HTML]{FFFFFF} 4} & \cellcolor[HTML]{3166FF}{\color[HTML]{FFFFFF} 5} & \cellcolor[HTML]{3166FF}{\color[HTML]{FFFFFF} 6} & \cellcolor[HTML]{3166FF}{\color[HTML]{FFFFFF} 7} & \cellcolor[HTML]{3166FF}{\color[HTML]{FFFFFF} 8} & \cellcolor[HTML]{3166FF}{\color[HTML]{FFFFFF} 9} & \cellcolor[HTML]{3166FF}{\color[HTML]{FFFFFF} 10} & \cellcolor[HTML]{3166FF}{\color[HTML]{FFFFFF} 11} & \cellcolor[HTML]{3166FF}{\color[HTML]{FFFFFF} 12} \\ \hline
\end{tabular}
\end{table}

\begin{itemize}
    \item Entrene una RNA compuesta con una capa intermedia con 2 perceptrones sigmoidales y una neurona en la salida también tipo sigmoidal. Use la regla BP. Proponga un conjunto de pesos para la red y un $\alpha$. Muestre el conjunto final de pesos final.
	\item Use la RNA entrenada para determinar la clase A o B de los puntos con coordenadas $\left(3,7\right)$ y $\left(6,8\right)$.
	\item Muestre que la capa interna de la RNA convierte el problema no lineal a uno lineal al mapear los puntos A y B en el espacio de las salidas de las 2 neuronas intermedias $s_1$,$s_2$.
\end{itemize}


\newpage
\subsection{Desarrollo de Sigmoide}
\begin{tcolorbox}[breakable, size=fbox, boxrule=1pt, pad at break*=1mm,colback=cellbackground, colframe=cellborder]
\prompt{In}{incolor}{1}{\boxspacing}
\begin{Verbatim}[commandchars=\\\{\}]
\PY{c+c1}{\PYZsh{} \PYZhy{}*\PYZhy{} coding: utf\PYZhy{}8 \PYZhy{}*\PYZhy{}}
\PY{k+kn}{import} \PY{n+nn}{numpy} \PY{k}{as} \PY{n+nn}{np}
\PY{k+kn}{import} \PY{n+nn}{matplotlib}\PY{n+nn}{.}\PY{n+nn}{pyplot} \PY{k}{as} \PY{n+nn}{plt}
 
\PY{c+c1}{\PYZsh{}\PYZpc{}\PYZpc{} Training patterns and targets}
\PY{n}{X} \PY{o}{=} \PY{n}{np}\PY{o}{.}\PY{n}{array}\PY{p}{(}\PY{p}{[}\PY{p}{[}\PY{l+m+mi}{2}\PY{p}{,} \PY{l+m+mi}{7}\PY{p}{]}\PY{p}{,} \PY{p}{[}\PY{l+m+mi}{3}\PY{p}{,} \PY{l+m+mi}{9}\PY{p}{]}\PY{p}{,} \PY{p}{[}\PY{l+m+mi}{4}\PY{p}{,} \PY{l+m+mi}{4}\PY{p}{]}\PY{p}{,} \PY{p}{[}\PY{l+m+mi}{5}\PY{p}{,} \PY{l+m+mi}{7}\PY{p}{]}\PY{p}{,} \PY{p}{[}\PY{l+m+mi}{6}\PY{p}{,} \PY{l+m+mi}{11}\PY{p}{]}\PY{p}{,} \PY{p}{[}\PY{l+m+mi}{7}\PY{p}{,} \PY{l+m+mi}{4}\PY{p}{]}\PY{p}{,} \PY{p}{[}\PY{l+m+mi}{8}\PY{p}{,}\PY{l+m+mi}{6}\PY{p}{]}\PY{p}{,} \PY{p}{[}\PY{l+m+mi}{8}\PY{p}{,}\PY{l+m+mi}{9}\PY{p}{]}\PY{p}{,} \PY{p}{[}\PY{l+m+mi}{10}\PY{p}{,} \PY{l+m+mi}{6}\PY{p}{]}\PY{p}{,} \PY{p}{[}\PY{l+m+mi}{10}\PY{p}{,} \PY{l+m+mi}{9}\PY{p}{]}\PY{p}{]}\PY{p}{)}
\PY{n}{t} \PY{o}{=} \PY{n}{np}\PY{o}{.}\PY{n}{array}\PY{p}{(}\PY{p}{[}\PY{l+m+mi}{0}\PY{p}{,} \PY{l+m+mi}{0}\PY{p}{,} \PY{l+m+mi}{0}\PY{p}{,} \PY{l+m+mi}{1}\PY{p}{,} \PY{l+m+mi}{0}\PY{p}{,} \PY{l+m+mi}{0}\PY{p}{,} \PY{l+m+mi}{1}\PY{p}{,} \PY{l+m+mi}{1}\PY{p}{,} \PY{l+m+mi}{1} \PY{p}{,}\PY{l+m+mi}{1}\PY{p}{]}\PY{p}{)}
\PY{n}{testinga}\PY{o}{=}\PY{p}{[}\PY{p}{]}

\PY{n}{alpha} \PY{o}{=} \PY{l+m+mf}{0.5}
\PY{n}{epochs} \PY{o}{=} \PY{l+m+mi}{9000}
 
\PY{c+c1}{\PYZsh{} Pesos iniciales}
\PY{n}{w1} \PY{o}{=} \PY{n}{np}\PY{o}{.}\PY{n}{array}\PY{p}{(}\PY{p}{[}\PY{p}{[}\PY{l+m+mf}{0.1}\PY{p}{,} \PY{l+m+mf}{0.2}\PY{p}{]}\PY{p}{,} \PY{p}{[}\PY{o}{\PYZhy{}}\PY{l+m+mf}{0.4}\PY{p}{,} \PY{l+m+mf}{0.5}\PY{p}{]}\PY{p}{]}\PY{p}{)} \PY{c+c1}{\PYZsh{} w\PYZus{}2}
\PY{n}{b1} \PY{o}{=} \PY{n}{np}\PY{o}{.}\PY{n}{array}\PY{p}{(}\PY{p}{[}\PY{l+m+mf}{0.3}\PY{p}{,} \PY{o}{\PYZhy{}}\PY{l+m+mf}{0.6}\PY{p}{]}\PY{p}{)} \PY{c+c1}{\PYZsh{} b\PYZus{}2}
\PY{n}{w2} \PY{o}{=} \PY{n}{np}\PY{o}{.}\PY{n}{array}\PY{p}{(}\PY{p}{[}\PY{l+m+mf}{0.7}\PY{p}{,} \PY{l+m+mf}{0.8}\PY{p}{]}\PY{p}{)} \PY{c+c1}{\PYZsh{} w\PYZus{}3}
\PY{n}{b2} \PY{o}{=} \PY{n}{np}\PY{o}{.}\PY{n}{array}\PY{p}{(}\PY{p}{[}\PY{o}{\PYZhy{}}\PY{l+m+mf}{0.9}\PY{p}{]}\PY{p}{)} \PY{c+c1}{\PYZsh{}b\PYZus{}3}
\end{Verbatim}
\end{tcolorbox}

    \begin{tcolorbox}[breakable, size=fbox, boxrule=1pt, pad at break*=1mm,colback=cellbackground, colframe=cellborder]
\prompt{In}{incolor}{2}{\boxspacing}
\begin{Verbatim}[commandchars=\\\{\}]
\PY{n}{err\PYZus{}vector} \PY{o}{=} \PY{p}{[}\PY{p}{]}
\PY{k}{for} \PY{n}{epoch} \PY{o+ow}{in} \PY{n+nb}{range}\PY{p}{(}\PY{n}{epochs}\PY{p}{)}\PY{p}{:}
    \PY{n}{count} \PY{o}{=} \PY{l+m+mi}{0}
    \PY{n}{err} \PY{o}{=} \PY{l+m+mi}{0}
    \PY{k}{for} \PY{n}{x} \PY{o+ow}{in} \PY{n}{X}\PY{p}{:}
        \PY{c+c1}{\PYZsh{} Feed forward}
        \PY{c+c1}{\PYZsh{}\PYZsh{}\PYZsh{}\PYZsh{}primera capa}
        \PY{n}{z1} \PY{o}{=} \PY{n}{np}\PY{o}{.}\PY{n}{dot}\PY{p}{(}\PY{n}{x}\PY{p}{,} \PY{n}{w1}\PY{o}{.}\PY{n}{T}\PY{p}{)} \PY{o}{+} \PY{n}{b1} 
        \PY{c+c1}{\PYZsh{} Sigmoid function}
        \PY{n}{a1} \PY{o}{=} \PY{l+m+mi}{1} \PY{o}{/} \PY{p}{(}\PY{l+m+mi}{1} \PY{o}{+} \PY{n}{np}\PY{o}{.}\PY{n}{exp}\PY{p}{(}\PY{o}{\PYZhy{}}\PY{n}{z1}\PY{p}{)}\PY{p}{)} \PY{c+c1}{\PYZsh{}a\PYZus{}2}
        \PY{c+c1}{\PYZsh{}\PYZsh{}\PYZsh{}\PYZsh{}segunda capa}
        \PY{n}{z2} \PY{o}{=} \PY{n}{np}\PY{o}{.}\PY{n}{dot}\PY{p}{(}\PY{n}{a1}\PY{p}{,} \PY{n}{w2}\PY{p}{)} \PY{o}{+} \PY{n}{b2}
        \PY{c+c1}{\PYZsh{} Sigmoid function}
        \PY{n}{a2} \PY{o}{=} \PY{l+m+mi}{1} \PY{o}{/} \PY{p}{(}\PY{l+m+mi}{1} \PY{o}{+} \PY{n}{np}\PY{o}{.}\PY{n}{exp}\PY{p}{(}\PY{o}{\PYZhy{}}\PY{n}{z2}\PY{p}{)}\PY{p}{)} \PY{c+c1}{\PYZsh{}a\PYZus{}3}
        
        
        \PY{c+c1}{\PYZsh{} Network error}
        \PY{n}{err} \PY{o}{+}\PY{o}{=} \PY{l+m+mf}{0.5} \PY{o}{*} \PY{n}{np}\PY{o}{.}\PY{n}{power}\PY{p}{(}\PY{n}{t}\PY{p}{[}\PY{n}{count}\PY{p}{]} \PY{o}{\PYZhy{}} \PY{n}{a2}\PY{p}{,} \PY{l+m+mi}{2}\PY{p}{)}
        
        \PY{c+c1}{\PYZsh{}\PYZsh{}\PYZsh{}\PYZsh{} Back propagation        }
        \PY{n}{L\PYZus{}error} \PY{o}{=} \PY{o}{\PYZhy{}}\PY{p}{(}\PY{n}{t}\PY{p}{[}\PY{n}{count}\PY{p}{]} \PY{o}{\PYZhy{}} \PY{n}{a2}\PY{p}{)} \PY{o}{*} \PY{n}{a2} \PY{o}{*} \PY{p}{(}\PY{l+m+mi}{1} \PY{o}{\PYZhy{}} \PY{n}{a2}\PY{p}{)}
        
        \PY{c+c1}{\PYZsh{} New output weights and bias}
        \PY{n}{n\PYZus{}w2} \PY{o}{=} \PY{n}{w2} \PY{o}{\PYZhy{}} \PY{n}{alpha} \PY{o}{*} \PY{n}{L\PYZus{}error} \PY{o}{*} \PY{n}{a1}
        \PY{n}{n\PYZus{}b2} \PY{o}{=} \PY{n}{b2} \PY{o}{\PYZhy{}} \PY{n}{alpha} \PY{o}{*} \PY{n}{L\PYZus{}error}
        
        \PY{c+c1}{\PYZsh{} Hidden layer error (l)}
        \PY{n}{l\PYZus{}error} \PY{o}{=} \PY{n}{L\PYZus{}error} \PY{o}{*} \PY{n}{w2} \PY{o}{*} \PY{n}{a1} \PY{o}{*} \PY{p}{(}\PY{l+m+mi}{1} \PY{o}{\PYZhy{}} \PY{n}{a1}\PY{p}{)}
        
        \PY{c+c1}{\PYZsh{} New hidden weights and bias}
        \PY{n}{n\PYZus{}b1} \PY{o}{=} \PY{n}{b1} \PY{o}{\PYZhy{}} \PY{n}{alpha} \PY{o}{*} \PY{n}{l\PYZus{}error}
        \PY{n}{x} \PY{o}{=} \PY{n}{np}\PY{o}{.}\PY{n}{reshape}\PY{p}{(}\PY{n}{x}\PY{p}{,} \PY{p}{(}\PY{l+m+mi}{1}\PY{p}{,} \PY{n+nb}{len}\PY{p}{(}\PY{n}{x}\PY{p}{)}\PY{p}{)}\PY{p}{)}
        \PY{n}{l\PYZus{}error} \PY{o}{=} \PY{n}{np}\PY{o}{.}\PY{n}{reshape}\PY{p}{(}\PY{n}{l\PYZus{}error}\PY{p}{,} \PY{p}{(}\PY{n+nb}{len}\PY{p}{(}\PY{n}{l\PYZus{}error}\PY{p}{)}\PY{p}{,} \PY{l+m+mi}{1}\PY{p}{)}\PY{p}{)}
        \PY{n}{n\PYZus{}w1} \PY{o}{=} \PY{n}{w1} \PY{o}{\PYZhy{}} \PY{n}{alpha} \PY{o}{*} \PY{n}{np}\PY{o}{.}\PY{n}{multiply}\PY{p}{(}\PY{n}{l\PYZus{}error}\PY{p}{,} \PY{n}{x}\PY{p}{)}
        
        \PY{c+c1}{\PYZsh{} Actualizacion de pesos y bias}
        \PY{n}{w1} \PY{o}{=} \PY{n}{n\PYZus{}w1}
        \PY{n}{b1} \PY{o}{=} \PY{n}{n\PYZus{}b1}
        \PY{n}{w2} \PY{o}{=} \PY{n}{n\PYZus{}w2}
        \PY{n}{b2} \PY{o}{=} \PY{n}{n\PYZus{}b2}
        
        \PY{n}{count} \PY{o}{+}\PY{o}{=} \PY{l+m+mi}{1}
        
    \PY{n}{err\PYZus{}vector}\PY{o}{.}\PY{n}{append}\PY{p}{(}\PY{n}{err} \PY{o}{/} \PY{n}{X}\PY{o}{.}\PY{n}{shape}\PY{p}{[}\PY{l+m+mi}{0}\PY{p}{]}\PY{p}{)}
    
\end{Verbatim}
\end{tcolorbox}

    \begin{tcolorbox}[breakable, size=fbox, boxrule=1pt, pad at break*=1mm,colback=cellbackground, colframe=cellborder]
\prompt{In}{incolor}{3}{\boxspacing}
\begin{Verbatim}[commandchars=\\\{\}]
\PY{c+c1}{\PYZsh{}\PYZpc{}\PYZpc{} Testing patterns}
\PY{n+nb}{print}\PY{p}{(}\PY{l+s+s1}{\PYZsq{}}\PY{l+s+s1}{       MLP result      }\PY{l+s+s1}{\PYZsq{}}\PY{p}{)}
\PY{n+nb}{print}\PY{p}{(}\PY{l+s+s1}{\PYZsq{}}\PY{l+s+s1}{Pat:          t:      out:}\PY{l+s+s1}{\PYZsq{}}\PY{p}{)}
\PY{n}{count} \PY{o}{=} \PY{l+m+mi}{0}
\PY{k}{for} \PY{n}{x} \PY{o+ow}{in} \PY{n}{X}\PY{p}{:}
    \PY{c+c1}{\PYZsh{} Feed forward}
    \PY{c+c1}{\PYZsh{}\PYZsh{}\PYZsh{}\PYZsh{}primera capa}
    \PY{n}{z1} \PY{o}{=} \PY{n}{np}\PY{o}{.}\PY{n}{dot}\PY{p}{(}\PY{n}{x}\PY{p}{,} \PY{n}{w1}\PY{o}{.}\PY{n}{T}\PY{p}{)} \PY{o}{+} \PY{n}{b1} 
    \PY{c+c1}{\PYZsh{} Sigmoid function}
    \PY{n}{a1} \PY{o}{=} \PY{l+m+mi}{1} \PY{o}{/} \PY{p}{(}\PY{l+m+mi}{1} \PY{o}{+} \PY{n}{np}\PY{o}{.}\PY{n}{exp}\PY{p}{(}\PY{o}{\PYZhy{}}\PY{n}{z1}\PY{p}{)}\PY{p}{)} 
    \PY{c+c1}{\PYZsh{}\PYZsh{}\PYZsh{}\PYZsh{}segunda capa}
    \PY{n}{z2} \PY{o}{=} \PY{n}{np}\PY{o}{.}\PY{n}{dot}\PY{p}{(}\PY{n}{a1}\PY{p}{,} \PY{n}{w2}\PY{p}{)} \PY{o}{+} \PY{n}{b2}
    \PY{c+c1}{\PYZsh{} Sigmoid function}
    \PY{n}{a2} \PY{o}{=} \PY{l+m+mi}{1} \PY{o}{/} \PY{p}{(}\PY{l+m+mi}{1} \PY{o}{+} \PY{n}{np}\PY{o}{.}\PY{n}{exp}\PY{p}{(}\PY{o}{\PYZhy{}}\PY{n}{z2}\PY{p}{)}\PY{p}{)}
    \PY{n+nb}{print}\PY{p}{(}\PY{l+s+s1}{\PYZsq{}}\PY{l+s+si}{\PYZob{}\PYZcb{}}\PY{l+s+s1}{. }\PY{l+s+si}{\PYZob{}\PYZcb{}}\PY{l+s+s1}{ \PYZhy{}\PYZhy{}\PYZhy{}\PYZhy{} }\PY{l+s+si}{\PYZob{}\PYZcb{}}\PY{l+s+s1}{ \PYZhy{}\PYZhy{}\PYZhy{}\PYZhy{}\PYZgt{} }\PY{l+s+si}{\PYZob{}:.3f\PYZcb{}}\PY{l+s+s1}{\PYZsq{}}\PY{o}{.}\PY{n}{format}\PY{p}{(}\PY{n}{count}\PY{p}{,} \PY{n}{x}\PY{p}{,} \PY{n}{t}\PY{p}{[}\PY{n}{count}\PY{p}{]}\PY{p}{,} \PY{n+nb}{float}\PY{p}{(}\PY{n}{a2}\PY{p}{)}\PY{p}{)}\PY{p}{)}
    \PY{n}{count} \PY{o}{+}\PY{o}{=} \PY{l+m+mi}{1}
    \PY{n}{testinga}\PY{o}{.}\PY{n}{append}\PY{p}{(}\PY{n}{a1}\PY{p}{)}
    
\PY{c+c1}{\PYZsh{}\PYZpc{}\PYZpc{}\PYZpc{}   }
\PY{c+c1}{\PYZsh{} Graph error}
 
\PY{n}{plt}\PY{o}{.}\PY{n}{figure}\PY{p}{(}\PY{l+m+mi}{0}\PY{p}{)}
\PY{n}{plt}\PY{o}{.}\PY{n}{plot}\PY{p}{(}\PY{n}{err\PYZus{}vector}\PY{p}{)}
\PY{n}{plt}\PY{o}{.}\PY{n}{xlabel}\PY{p}{(}\PY{l+s+s1}{\PYZsq{}}\PY{l+s+s1}{Epocas}\PY{l+s+s1}{\PYZsq{}}\PY{p}{)}
\PY{n}{plt}\PY{o}{.}\PY{n}{ylabel}\PY{p}{(}\PY{l+s+s1}{\PYZsq{}}\PY{l+s+s1}{Error}\PY{l+s+s1}{\PYZsq{}}\PY{p}{)}
\PY{n}{plt}\PY{o}{.}\PY{n}{title}\PY{p}{(}\PY{l+s+s1}{\PYZsq{}}\PY{l+s+s1}{BP algorithm}\PY{l+s+s1}{\PYZsq{}}\PY{p}{)}
\PY{n}{plt}\PY{o}{.}\PY{n}{show}\PY{p}{(}\PY{p}{)}
\end{Verbatim}
\end{tcolorbox}

    \begin{Verbatim}[commandchars=\\\{\}]
       MLP result
Pat:          t:      out:
0. [2 7] ---- 0 ----> 0.013
1. [3 9] ---- 0 ----> 0.015
2. [4 4] ---- 0 ----> 0.016
3. [5 7] ---- 1 ----> 0.986
4. [ 6 11] ---- 0 ----> 0.016
5. [7 4] ---- 0 ----> 0.021
6. [8 6] ---- 1 ----> 0.982
7. [8 9] ---- 1 ----> 0.990
8. [10  6] ---- 1 ----> 0.985
9. [10  9] ---- 1 ----> 0.990
    \end{Verbatim}

    \begin{center}
    \adjustimage{max size={0.9\linewidth}{0.9\paperheight}}{sections/RNACompuesta/output_2_1.png}
    \end{center}
    { \hspace*{\fill} \\}
    
    \begin{tcolorbox}[breakable, size=fbox, boxrule=1pt, pad at break*=1mm,colback=cellbackground, colframe=cellborder]
\prompt{In}{incolor}{4}{\boxspacing}
\begin{Verbatim}[commandchars=\\\{\}]
\PY{c+c1}{\PYZsh{} Decision boundaries}
\PY{c+c1}{\PYZsh{}def dec\PYZus{}boundaries(X, t, w\PYZus{}2, b\PYZus{}2, w\PYZus{}3, b\PYZus{}3):}
    
\PY{c+c1}{\PYZsh{} Creating mesh}
\PY{n}{h} \PY{o}{=} \PY{l+m+mf}{0.01}
\PY{n}{x\PYZus{}min}\PY{p}{,} \PY{n}{x\PYZus{}max} \PY{o}{=} \PY{o}{\PYZhy{}}\PY{l+m+mf}{0.2}\PY{p}{,} \PY{l+m+mf}{12.2}
\PY{n}{y\PYZus{}min}\PY{p}{,} \PY{n}{y\PYZus{}max} \PY{o}{=} \PY{o}{\PYZhy{}}\PY{l+m+mf}{0.2}\PY{p}{,} \PY{l+m+mf}{15.2}
 
\PY{n}{xx}\PY{p}{,} \PY{n}{yy} \PY{o}{=} \PY{n}{np}\PY{o}{.}\PY{n}{meshgrid}\PY{p}{(}\PY{n}{np}\PY{o}{.}\PY{n}{arange}\PY{p}{(}\PY{n}{x\PYZus{}min}\PY{p}{,} \PY{n}{x\PYZus{}max}\PY{p}{,} \PY{n}{h}\PY{p}{)}\PY{p}{,}\PY{n}{np}\PY{o}{.}\PY{n}{arange}\PY{p}{(}\PY{n}{y\PYZus{}min}\PY{p}{,} \PY{n}{y\PYZus{}max}\PY{p}{,} \PY{n}{h}\PY{p}{)}\PY{p}{)}
 
\PY{n}{Z} \PY{o}{=} \PY{n}{np}\PY{o}{.}\PY{n}{c\PYZus{}}\PY{p}{[}\PY{n}{xx}\PY{o}{.}\PY{n}{ravel}\PY{p}{(}\PY{p}{)}\PY{p}{,} \PY{n}{yy}\PY{o}{.}\PY{n}{ravel}\PY{p}{(}\PY{p}{)}\PY{p}{]}
\PY{n}{out} \PY{o}{=} \PY{n}{np}\PY{o}{.}\PY{n}{zeros}\PY{p}{(}\PY{n}{np}\PY{o}{.}\PY{n}{shape}\PY{p}{(}\PY{n}{Z}\PY{p}{)}\PY{p}{[}\PY{l+m+mi}{0}\PY{p}{]}\PY{p}{)}
 
\PY{c+c1}{\PYZsh{}\PYZpc{}\PYZpc{} Output model}
\PY{k}{for} \PY{n}{i} \PY{o+ow}{in} \PY{n+nb}{range}\PY{p}{(}\PY{n+nb}{len}\PY{p}{(}\PY{n}{out}\PY{p}{)}\PY{p}{)}\PY{p}{:}
    \PY{c+c1}{\PYZsh{} Feed forward}
    \PY{c+c1}{\PYZsh{}\PYZsh{}\PYZsh{}\PYZsh{}primera capa}
    \PY{n}{z1} \PY{o}{=} \PY{n}{np}\PY{o}{.}\PY{n}{dot}\PY{p}{(}\PY{n}{Z}\PY{p}{[}\PY{n}{i}\PY{p}{]}\PY{p}{,} \PY{n}{w1}\PY{o}{.}\PY{n}{T}\PY{p}{)} \PY{o}{+} \PY{n}{b1} 
    \PY{c+c1}{\PYZsh{} Sigmoid function}
    \PY{n}{a1} \PY{o}{=} \PY{l+m+mi}{1} \PY{o}{/} \PY{p}{(}\PY{l+m+mi}{1} \PY{o}{+} \PY{n}{np}\PY{o}{.}\PY{n}{exp}\PY{p}{(}\PY{o}{\PYZhy{}}\PY{n}{z1}\PY{p}{)}\PY{p}{)} 
    \PY{c+c1}{\PYZsh{}\PYZsh{}\PYZsh{}\PYZsh{}segunda capa}
    \PY{n}{z2} \PY{o}{=} \PY{n}{np}\PY{o}{.}\PY{n}{dot}\PY{p}{(}\PY{n}{a1}\PY{p}{,} \PY{n}{w2}\PY{p}{)} \PY{o}{+} \PY{n}{b2}
    \PY{c+c1}{\PYZsh{} Sigmoid function}
    \PY{n}{a2} \PY{o}{=} \PY{l+m+mi}{1} \PY{o}{/} \PY{p}{(}\PY{l+m+mi}{1} \PY{o}{+} \PY{n}{np}\PY{o}{.}\PY{n}{exp}\PY{p}{(}\PY{o}{\PYZhy{}}\PY{n}{z2}\PY{p}{)}\PY{p}{)}
    \PY{n}{out}\PY{p}{[}\PY{n}{i}\PY{p}{]} \PY{o}{=} \PY{n}{a2}
    
\PY{c+c1}{\PYZsh{} out = (out \PYZgt{}= 0.5).astype(int)}
\PY{n}{out} \PY{o}{=} \PY{n}{out}\PY{o}{.}\PY{n}{reshape}\PY{p}{(}\PY{n}{xx}\PY{o}{.}\PY{n}{shape}\PY{p}{)}
\PY{n}{levels} \PY{o}{=} \PY{n}{np}\PY{o}{.}\PY{n}{linspace}\PY{p}{(}\PY{l+m+mi}{0}\PY{p}{,} \PY{l+m+mi}{1}\PY{p}{)}
\PY{n}{plt}\PY{o}{.}\PY{n}{figure}\PY{p}{(}\PY{l+m+mi}{1}\PY{p}{)}
\PY{n}{plt}\PY{o}{.}\PY{n}{contourf}\PY{p}{(}\PY{n}{xx}\PY{p}{,} \PY{n}{yy}\PY{p}{,} \PY{n}{out}\PY{p}{,} \PY{n}{levels}\PY{p}{)}
\PY{n}{plt}\PY{o}{.}\PY{n}{colorbar}\PY{p}{(}\PY{p}{)}
 
\PY{c+c1}{\PYZsh{} Plotting data}
\PY{n}{lis} \PY{o}{=} \PY{n}{np}\PY{o}{.}\PY{n}{unique}\PY{p}{(}\PY{n}{t}\PY{p}{)}
\PY{k}{for} \PY{n}{i} \PY{o+ow}{in} \PY{n+nb}{range}\PY{p}{(}\PY{n+nb}{len}\PY{p}{(}\PY{n}{t}\PY{p}{)}\PY{p}{)}\PY{p}{:}
    \PY{k}{if} \PY{n}{i} \PY{o}{==} \PY{l+m+mi}{0}\PY{p}{:}
        \PY{n}{pos} \PY{o}{=} \PY{n}{np}\PY{o}{.}\PY{n}{where}\PY{p}{(}\PY{n}{t} \PY{o}{==} \PY{l+m+mi}{0}\PY{p}{)}\PY{p}{[}\PY{l+m+mi}{0}\PY{p}{]}
        \PY{n}{plt}\PY{o}{.}\PY{n}{plot}\PY{p}{(}\PY{n}{X}\PY{p}{[}\PY{n}{pos}\PY{p}{]}\PY{p}{[}\PY{p}{:}\PY{p}{,} \PY{l+m+mi}{0}\PY{p}{]}\PY{p}{,} \PY{n}{X}\PY{p}{[}\PY{n}{pos}\PY{p}{]}\PY{p}{[}\PY{p}{:}\PY{p}{,} \PY{l+m+mi}{1}\PY{p}{]}\PY{p}{,} \PY{l+s+s1}{\PYZsq{}}\PY{l+s+s1}{o}\PY{l+s+s1}{\PYZsq{}}\PY{p}{,} \PY{n}{color} \PY{o}{=} \PY{l+s+s1}{\PYZsq{}}\PY{l+s+s1}{m}\PY{l+s+s1}{\PYZsq{}}\PY{p}{,} \PY{n}{markersize} \PY{o}{=} \PY{l+m+mi}{15}\PY{p}{)}
    \PY{k}{else}\PY{p}{:}
        \PY{n}{pos} \PY{o}{=} \PY{n}{np}\PY{o}{.}\PY{n}{where}\PY{p}{(}\PY{n}{t} \PY{o}{==} \PY{l+m+mi}{1}\PY{p}{)}\PY{p}{[}\PY{l+m+mi}{0}\PY{p}{]}
        \PY{n}{plt}\PY{o}{.}\PY{n}{plot}\PY{p}{(}\PY{n}{X}\PY{p}{[}\PY{n}{pos}\PY{p}{]}\PY{p}{[}\PY{p}{:}\PY{p}{,} \PY{l+m+mi}{0}\PY{p}{]}\PY{p}{,} \PY{n}{X}\PY{p}{[}\PY{n}{pos}\PY{p}{]}\PY{p}{[}\PY{p}{:}\PY{p}{,} \PY{l+m+mi}{1}\PY{p}{]}\PY{p}{,} \PY{l+s+s1}{\PYZsq{}}\PY{l+s+s1}{x}\PY{l+s+s1}{\PYZsq{}}\PY{p}{,} \PY{n}{color} \PY{o}{=} \PY{l+s+s1}{\PYZsq{}}\PY{l+s+s1}{r}\PY{l+s+s1}{\PYZsq{}}\PY{p}{,} \PY{n}{markersize} \PY{o}{=} \PY{l+m+mi}{15}\PY{p}{)}
 
\PY{n}{plt}\PY{o}{.}\PY{n}{title}\PY{p}{(}\PY{l+s+s1}{\PYZsq{}}\PY{l+s+s1}{Decision boundary}\PY{l+s+s1}{\PYZsq{}}\PY{p}{)}
\PY{n}{plt}\PY{o}{.}\PY{n}{xlabel}\PY{p}{(}\PY{l+s+s1}{\PYZsq{}}\PY{l+s+s1}{x}\PY{l+s+s1}{\PYZsq{}}\PY{p}{)}
\PY{n}{plt}\PY{o}{.}\PY{n}{ylabel}\PY{p}{(}\PY{l+s+s1}{\PYZsq{}}\PY{l+s+s1}{y}\PY{l+s+s1}{\PYZsq{}}\PY{p}{)}
\PY{n}{plt}\PY{o}{.}\PY{n}{show}\PY{p}{(}\PY{p}{)}
\end{Verbatim}
\end{tcolorbox}

    \begin{center}
    \adjustimage{max size={0.9\linewidth}{0.9\paperheight}}{sections/RNACompuesta/output_3_0.png}
    \end{center}
    { \hspace*{\fill} \\}
    
    \begin{tcolorbox}[breakable, size=fbox, boxrule=1pt, pad at break*=1mm,colback=cellbackground, colframe=cellborder]
\prompt{In}{incolor}{5}{\boxspacing}
\begin{Verbatim}[commandchars=\\\{\}]
\PY{c+c1}{\PYZsh{} out = (out \PYZgt{}= 0.5).astype(int)}
\PY{n}{out} \PY{o}{=} \PY{n}{out}\PY{o}{.}\PY{n}{reshape}\PY{p}{(}\PY{n}{xx}\PY{o}{.}\PY{n}{shape}\PY{p}{)}
\PY{n}{levels} \PY{o}{=} \PY{n}{np}\PY{o}{.}\PY{n}{linspace}\PY{p}{(}\PY{l+m+mi}{0}\PY{p}{,} \PY{l+m+mi}{1}\PY{p}{)}
\PY{n}{plt}\PY{o}{.}\PY{n}{figure}\PY{p}{(}\PY{l+m+mi}{2}\PY{p}{)}
\PY{n}{plt}\PY{o}{.}\PY{n}{contourf}\PY{p}{(}\PY{n}{xx}\PY{p}{,} \PY{n}{yy}\PY{p}{,} \PY{n}{out}\PY{p}{,} \PY{n}{levels}\PY{p}{)}
\PY{n}{plt}\PY{o}{.}\PY{n}{colorbar}\PY{p}{(}\PY{p}{)}
 
\PY{c+c1}{\PYZsh{} Plotting data}
\PY{n}{plt}\PY{o}{.}\PY{n}{plot}\PY{p}{(}\PY{l+m+mi}{3}\PY{p}{,} \PY{l+m+mi}{7}\PY{p}{,} \PY{l+s+s1}{\PYZsq{}}\PY{l+s+s1}{o}\PY{l+s+s1}{\PYZsq{}}\PY{p}{,} \PY{n}{color} \PY{o}{=} \PY{l+s+s1}{\PYZsq{}}\PY{l+s+s1}{m}\PY{l+s+s1}{\PYZsq{}}\PY{p}{,} \PY{n}{markersize} \PY{o}{=} \PY{l+m+mi}{15}\PY{p}{)}
\PY{n}{plt}\PY{o}{.}\PY{n}{plot}\PY{p}{(}\PY{l+m+mi}{6}\PY{p}{,} \PY{l+m+mi}{8}\PY{p}{,} \PY{l+s+s1}{\PYZsq{}}\PY{l+s+s1}{x}\PY{l+s+s1}{\PYZsq{}}\PY{p}{,} \PY{n}{color} \PY{o}{=} \PY{l+s+s1}{\PYZsq{}}\PY{l+s+s1}{r}\PY{l+s+s1}{\PYZsq{}}\PY{p}{,} \PY{n}{markersize} \PY{o}{=} \PY{l+m+mi}{15}\PY{p}{)}
\PY{n}{plt}\PY{o}{.}\PY{n}{title}\PY{p}{(}\PY{l+s+s1}{\PYZsq{}}\PY{l+s+s1}{Prueba final Decision boundary}\PY{l+s+s1}{\PYZsq{}}\PY{p}{)}
\PY{n}{plt}\PY{o}{.}\PY{n}{xlabel}\PY{p}{(}\PY{l+s+s1}{\PYZsq{}}\PY{l+s+s1}{x}\PY{l+s+s1}{\PYZsq{}}\PY{p}{)}
\PY{n}{plt}\PY{o}{.}\PY{n}{ylabel}\PY{p}{(}\PY{l+s+s1}{\PYZsq{}}\PY{l+s+s1}{y}\PY{l+s+s1}{\PYZsq{}}\PY{p}{)}
\PY{n}{plt}\PY{o}{.}\PY{n}{show}\PY{p}{(}\PY{p}{)}
\end{Verbatim}
\end{tcolorbox}

    \begin{center}
    \adjustimage{max size={0.9\linewidth}{0.9\paperheight}}{sections/RNACompuesta/output_4_0.png}
    \end{center}
    { \hspace*{\fill} \\}
    
    \begin{tcolorbox}[breakable, size=fbox, boxrule=1pt, pad at break*=1mm,colback=cellbackground, colframe=cellborder]
\prompt{In}{incolor}{6}{\boxspacing}
\begin{Verbatim}[commandchars=\\\{\}]
\PY{c+c1}{\PYZsh{}print(testinga)}
\end{Verbatim}
\end{tcolorbox}

    \begin{tcolorbox}[breakable, size=fbox, boxrule=1pt, pad at break*=1mm,colback=cellbackground, colframe=cellborder]
\prompt{In}{incolor}{7}{\boxspacing}
\begin{Verbatim}[commandchars=\\\{\}]
\PY{n}{plt}\PY{o}{.}\PY{n}{figure}\PY{p}{(}\PY{l+m+mi}{3}\PY{p}{)}

\PY{n}{plt}\PY{o}{.}\PY{n}{xlim}\PY{p}{(}\PY{n}{xmin} \PY{o}{=} \PY{o}{\PYZhy{}}\PY{l+m+mf}{0.1}\PY{p}{,} \PY{n}{xmax}\PY{o}{=}\PY{l+m+mf}{1.1}\PY{p}{)}
\PY{n}{plt}\PY{o}{.}\PY{n}{ylim}\PY{p}{(}\PY{n}{ymin} \PY{o}{=} \PY{o}{\PYZhy{}}\PY{l+m+mf}{0.1}\PY{p}{,} \PY{n}{ymax}\PY{o}{=}\PY{l+m+mf}{1.1}\PY{p}{)}
\PY{n}{lis} \PY{o}{=} \PY{n}{np}\PY{o}{.}\PY{n}{unique}\PY{p}{(}\PY{n}{testinga}\PY{p}{)}
\PY{k}{for} \PY{n}{i} \PY{o+ow}{in} \PY{n+nb}{range}\PY{p}{(}\PY{n+nb}{len}\PY{p}{(}\PY{n}{testinga}\PY{p}{)}\PY{p}{)}\PY{p}{:}
    \PY{k}{if} \PY{n}{testinga}\PY{p}{[}\PY{n}{i}\PY{p}{]}\PY{p}{[}\PY{l+m+mi}{0}\PY{p}{]} \PY{o}{\PYZgt{}}\PY{o}{=} \PY{l+m+mf}{0.5} \PY{o+ow}{and} \PY{n}{testinga}\PY{p}{[}\PY{n}{i}\PY{p}{]}\PY{p}{[}\PY{l+m+mi}{1}\PY{p}{]} \PY{o}{\PYZgt{}}\PY{o}{=} \PY{l+m+mf}{0.5}\PY{p}{:}
        \PY{n}{plt}\PY{o}{.}\PY{n}{plot}\PY{p}{(} \PY{n}{testinga}\PY{p}{[}\PY{n}{i}\PY{p}{]}\PY{p}{[}\PY{l+m+mi}{0}\PY{p}{]}\PY{p}{,} \PY{n}{testinga}\PY{p}{[}\PY{n}{i}\PY{p}{]}\PY{p}{[}\PY{l+m+mi}{1}\PY{p}{]}\PY{p}{,} \PY{l+s+s1}{\PYZsq{}}\PY{l+s+s1}{o}\PY{l+s+s1}{\PYZsq{}}\PY{p}{,} \PY{n}{color} \PY{o}{=} \PY{l+s+s1}{\PYZsq{}}\PY{l+s+s1}{m}\PY{l+s+s1}{\PYZsq{}}\PY{p}{,} \PY{n}{markersize} \PY{o}{=} \PY{l+m+mi}{15}\PY{p}{,} \PY{n}{alpha}\PY{o}{=}\PY{l+m+mf}{0.5}\PY{p}{)}
    \PY{k}{else}\PY{p}{:}
        \PY{n}{plt}\PY{o}{.}\PY{n}{plot}\PY{p}{(} \PY{n}{testinga}\PY{p}{[}\PY{n}{i}\PY{p}{]}\PY{p}{[}\PY{l+m+mi}{0}\PY{p}{]}\PY{p}{,} \PY{n}{testinga}\PY{p}{[}\PY{n}{i}\PY{p}{]}\PY{p}{[}\PY{l+m+mi}{1}\PY{p}{]}\PY{p}{,} \PY{l+s+s1}{\PYZsq{}}\PY{l+s+s1}{x}\PY{l+s+s1}{\PYZsq{}}\PY{p}{,} \PY{n}{color} \PY{o}{=} \PY{l+s+s1}{\PYZsq{}}\PY{l+s+s1}{b}\PY{l+s+s1}{\PYZsq{}}\PY{p}{,} \PY{n}{markersize} \PY{o}{=} \PY{l+m+mi}{15}\PY{p}{,} \PY{n}{alpha}\PY{o}{=}\PY{l+m+mf}{0.5}\PY{p}{)}
\PY{n}{x\PYZus{}map} \PY{o}{=} \PY{n}{np}\PY{o}{.}\PY{n}{linspace}\PY{p}{(}\PY{o}{\PYZhy{}}\PY{l+m+mf}{0.1}\PY{p}{,} \PY{l+m+mf}{1.1}\PY{p}{)}
\PY{n}{y\PYZus{}map} \PY{o}{=} \PY{o}{\PYZhy{}}\PY{p}{(}\PY{n}{w2}\PY{p}{[}\PY{l+m+mi}{0}\PY{p}{]} \PY{o}{/} \PY{n}{w2}\PY{p}{[}\PY{l+m+mi}{1}\PY{p}{]}\PY{p}{)} \PY{o}{*} \PY{p}{(}\PY{n}{x\PYZus{}map}\PY{p}{)}\PY{o}{\PYZhy{}}\PY{p}{(}\PY{n}{b2}\PY{o}{/}\PY{n}{w2}\PY{p}{[}\PY{l+m+mi}{1}\PY{p}{]}\PY{p}{)}

\PY{n}{plt}\PY{o}{.}\PY{n}{plot}\PY{p}{(}\PY{n}{x\PYZus{}map}\PY{p}{,} \PY{n}{y\PYZus{}map}\PY{p}{,} \PY{n}{color} \PY{o}{=} \PY{l+s+s1}{\PYZsq{}}\PY{l+s+s1}{r}\PY{l+s+s1}{\PYZsq{}}\PY{p}{,} \PY{n}{linewidth}\PY{o}{=}\PY{l+m+mi}{2}\PY{p}{)}
\PY{n}{plt}\PY{o}{.}\PY{n}{title}\PY{p}{(}\PY{l+s+s1}{\PYZsq{}}\PY{l+s+s1}{Decision boundary}\PY{l+s+s1}{\PYZsq{}}\PY{p}{)}
\PY{n}{plt}\PY{o}{.}\PY{n}{xlabel}\PY{p}{(}\PY{l+s+s1}{\PYZsq{}}\PY{l+s+s1}{x}\PY{l+s+s1}{\PYZsq{}}\PY{p}{)}
\PY{n}{plt}\PY{o}{.}\PY{n}{ylabel}\PY{p}{(}\PY{l+s+s1}{\PYZsq{}}\PY{l+s+s1}{y}\PY{l+s+s1}{\PYZsq{}}\PY{p}{)}
\PY{n}{plt}\PY{o}{.}\PY{n}{show}\PY{p}{(}\PY{p}{)}
\end{Verbatim}
\end{tcolorbox}

    \begin{center}
    \adjustimage{max size={0.9\linewidth}{0.9\paperheight}}{sections/RNACompuesta/output_6_0.png}
    \end{center}
    { \hspace*{\fill} \\}
    

\clearpage